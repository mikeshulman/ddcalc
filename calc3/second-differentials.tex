\documentclass[12pt]{amsart}
\usepackage{fullpage}
\makeatletter

%% AMS
\usepackage{amsmath,amssymb,latexsym}

%% Theorems
\usepackage{xcolor}
\definecolor{darkgreen}{rgb}{0,0.45,0} 
\usepackage[pagebackref,colorlinks,citecolor=darkgreen,linkcolor=darkgreen]{hyperref}
\usepackage{cleveref,aliascnt}
\usepackage{amsthm}
\def\defthm#1#2#3{%
  %% Ensure all theorem types are numbered with the same counter
  \newaliascnt{#1}{thm}
  \newtheorem{#1}[#1]{#2}
  \aliascntresetthe{#1}
  %% Tell cleveref's \cref what to call things
  \crefname{#1}{#2}{#3}}
\newtheorem{thm}{Theorem}[section]
\crefname{thm}{Theorem}{Theorems}
\theoremstyle{definition}
\defthm{defn}{Definition}{Definitions}
%\theoremstyle{remark}
\defthm{rmk}{Remark}{Remarks}
\defthm{advrmk}{Advanced Remark}{Advanced Remarks}
\newenvironment{adv}{\small\begin{advrmk}}{\end{advrmk}}
\defthm{eg}{Example}{Examples}
\defthm{egs}{Examples}{Examples}
\defthm{sage}{Sage}{Sages}

%% Reference format for sections
\crefformat{section}{\S#2#1#3}
\Crefformat{section}{Section~#2#1#3}
\crefrangeformat{section}{\S\S#3#1#4--#5#2#6}
\Crefrangeformat{section}{Sections~#3#1#4--#5#2#6}
\crefmultiformat{section}{\S\S#2#1#3}{ and~#2#1#3}{, #2#1#3}{ and~#2#1#3}
\Crefmultiformat{section}{Sections~#2#1#3}{ and~#2#1#3}{, #2#1#3}{ and~#2#1#3}
\crefrangemultiformat{section}{\S\S#3#1#4--#5#2#6}{ and~#3#1#4--#5#2#6}{, #3#1#4--#5#2#6}{ and~#3#1#4--#5#2#6}
\Crefrangemultiformat{section}{Sections~#3#1#4--#5#2#6}{ and~#3#1#4--#5#2#6}{, #3#1#4--#5#2#6}{ and~#3#1#4--#5#2#6}

%% Use the theorem counter for equations as well
\let\c@equation\c@thm
\numberwithin{equation}{section}

%% Pictures
\usepackage{tikz}
\usepackage{graphicx}

%% Differentials
\newcommand{\dd}{\ensuremath{\mathrm{d}}}
\newcommand{\ed}{\ensuremath{\mathrm{d}\wedge}}

%% Differentials of variables
\newcommand{\dx}{{\ensuremath{\dd x}}}
\newcommand{\dy}{{\ensuremath{\dd y}}}
\newcommand{\dz}{{\ensuremath{\dd z}}}
\newcommand{\dzbar}{{\ensuremath{\dd \bar{z}}}}
\newcommand{\dt}{{\ensuremath{\dd t}}}
\newcommand{\du}{{\ensuremath{\dd u}}}
\newcommand{\dv}{{\ensuremath{\dd v}}}
\newcommand{\df}{{\ensuremath{\dd f}}}
\newcommand{\dg}{{\ensuremath{\dd g}}}
\renewcommand{\dh}{{\ensuremath{\dd h}}}
\newcommand{\dr}{{\ensuremath{\dd r}}}
\newcommand{\drho}{{\ensuremath{\dd \rho}}}
\newcommand{\dtheta}{{\ensuremath{\dd \theta}}}
\newcommand{\dphi}{{\ensuremath{\dd \phi}}}
\newcommand{\dpx}{{\ensuremath{\dd \pt{x}}}}

%% Not differentials of anything
\usepackage[T1]{fontenc}
\usepackage{lmodern}
\newcommand{\dl}{\ensuremath{\mathrm{\dj}\ell}}
\newcommand{\dA}{\ensuremath{\mathrm{\dj}A}}
\newcommand{\dV}{\ensuremath{\mathrm{\dj}V}}
\newcommand{\dn}{\ensuremath{\mathrm{\dj}\vc{n}}}

%% Wedge product
%\newcommand{\wedge}{\wedge}

%% Partial derivatives
% \newcommand{\pder}[2]{\frac{\partial #1}{\partial #2}}
% \newcommand{\pdertwo}[2]{\frac{\partial^2 #1}{\partial #2^2}}
% \newcommand{\pdermixed}[3]{\frac{\partial^2 #1}{\partial #2 \,\partial #3}}
\newcommand{\pder}[2]{{#1}_{#2}}
\newcommand{\pdertwo}[2]{{#1}_{{#2}{#2}}}
\newcommand{\pdermixed}[3]{{#1}_{{#2}{#3}}}

%% Transposing vectors into forms
\newcommand{\tr}[1]{{#1}^\top}
\newcommand{\ptr}[1]{{(#1)}^\top}

%% Hodge star on forms
\newcommand{\str}[1]{{#1}^*}
\newcommand{\pstr}[1]{{(#1)}^*}

%% Points and vectors
\newcommand{\pt}[1]{\mathbf{#1}}
\newcommand{\ptc}[1]{(#1)}
\newcommand{\ptx}{\pt{x}}
% \newcommand{\vc}[1]{\vec{#1}}
\newcommand{\vc}[1]{\mathbf{#1}}
\newcommand{\vcc}[1]{(#1)}
\newcommand{\lrvcc}[1]{\left(#1\right)}

%% Vector operations
\newcommand{\mgn}[1]{\Vert #1 \Vert}
%\newcommand{\dotp}[2]{\langle #1 , #2 \rangle}
\newcommand{\dotp}[2]{#1 \cdot #2}
\newcommand{\crossp}[2]{#1 \times #2}

%% Vector calculus operations
\newcommand{\grad}[1]{\nabla #1}
\renewcommand{\div}[1]{\nabla \cdot #1}
\newcommand{\curl}[1]{\nabla \times #1}

%% Fractions
\newcommand{\half}{\ensuremath{\textstyle\frac{1}{2}}}
\newcommand{\halfi}{\ensuremath{\textstyle\frac{1}{2i}}}
\newcommand{\halfpii}{\ensuremath{\textstyle\frac{1}{2\pi i}}}
\newcommand{\third}{\ensuremath{\textstyle\frac{1}{3}}}

%% Orders
\renewcommand{\th}{^{\mathrm{th}}}
\newcommand{\st}{^{\mathrm{st}}}

%% Approximation to specified order
\newcommand{\apx}[1]{\mathrel{\overset{#1}{\approx}}}

%% Riemann sums
\newcommand{\rstag}[2]{{#1}_{#2}^*}

%% Complex conjugation
\newcommand{\conj}[1]{\bar{#1}}
\newcommand{\zbar}{\conj{z}}
\newcommand{\wbar}{\conj{w}}

%% Parametrized curves
\newcommand{\curve}{C}
\newcommand{\curvept}{\mathbf{x}}
\newcommand{\dcurvept}{\dd\mathbf{x}}
\newcommand{\curvex}{x}
\newcommand{\curvey}{y}
\newcommand{\curvez}{z}

%% Regions
\newcommand{\region}{R}
\newcommand{\bdry}{\partial}

%% Parametrized surfaces
\newcommand{\surf}{S}
\newcommand{\surfx}{x}
\newcommand{\surfy}{y}
\newcommand{\surfz}{z}

%% Integrals of differential forms.  The only argument is the
%% integration manifold; the differential form just comes afterwards.
\newcommand{\lint}[1]{\int_{#1}} % line integral over a curve
\newcommand{\sint}[1]{\iint_{#1}} % surface integral form over a surface
\newcommand{\vint}[1]{\iiint_{#1}} % volume integral over a spatial region

%% "Classical" integrals.  For these the integrand is a *function*
%% (well, technically, an expression with free variables) which is
%% given as a second argument.
\newcommand{\dint}[2]{\iint_{#1} #2 \,\dA} % double integral over a plane region
\newcommand{\tint}[2]{\iiint_{#1} #2 \,\dV} % triple integral over a spatial region
\newcommand{\itint}[4]{\int_{#2}^{#3} #4 \,\dd #1} % ordinary one-variable integral over an interval
\newcommand{\itdint}[7]{\int_{#2}^{#3} \int_{#5}^{#6} #7 \,\dd #4\,\dd #1} % double iterated integral
% Can't have more than 9 parameters
%\newcommand{\ittint}[10]{\int_{#2}^{#3} \int_{#5}^{#6} \int_{#8}^{#9} #10 \,\dd #7\,\dd #4\,\dd #1} % triple iterated integral
\makeatletter
\newcommand{\ittint}[3]{\int_{#2}^{#3} \@ittint #1}
\newcommand{\@ittint}[8]{\int_{#3}^{#4} \int_{#6}^{#7} #8 \,\dd #5\,\dd #2\,\dd #1} % triple iterated integral
\makeatother

%% Textbooks
\usepackage{version}
\includeversion{notextbook}
\excludeversion{stewart}
\excludeversion{hugheshallett}
\excludeversion{rogawski}

\makeatother

\title{Second differentials}
\begin{document}
\maketitle

In one-variable calculus, you probably had occasion to consider not only the derivative $f'$ of a function $f$, but occasionally its \emph{second} and higher derivatives $f''$, $f'''$, and so on, which are obtained by differentiating the first derivative further.
Let's consider how these can be expressed using differentials, and how they generalize to the multivariable case.

\section{Second differentials}
\label{sec:second-differentials}

Let's consider first the case when we start with a one-variable function $f$.
Then its differential $\df$ is a function of \emph{two} variables, $x$ and $\dx$.
Thus, if we want to take the differential of $\df$ again, we end up already in the multivariable situation.

\begin{eg}
  Suppose $f(x) = x^4$.  Then $\df = 4x^3 \,\dx$, and therefore
  \begin{align*}
    \dd(\df) &= \dd(4x^3 \,\dx)\\
    &= \dx \;\dd(4x^3) + 4x^3 \, \dd(\dx) \qquad\text{(by the product rule)}\\
    &= \dx \,(12x^2\,\dx) + 4x^3 \, \dd(\dx)\\
    &= 12x^2\,\dx^2 + 4x^3 \, \dd(\dx).
  \end{align*}
\end{eg}

We write $\dd(\dx)$ as $\dd^2x$, and similarly for other variables and functions.
Thus, in this example we have $\dd^2f = 12x^2\,\dx^2 + 4x^3 \, \dd^2x$.
Note that $12x^2$ is the second derivative of the original function $f$, while $4x^3$ is its first derivative.
An analogous formula is true in general: since $\df = f'(x) \,\dx$, we have
\begin{align}
  \dd^2f &= \dd(f'(x)\,\dx) \notag\\
  &= \dx \;\dd(f'(x)) + f'(x) \, \dd(\dx) \notag\\
  &= \dx \,(f''(x)\, \dx) + f'(x) \, \dd^2x \notag\\
  &= f''(x) \,\dx^2 + f'(x) \,\dd^2x.\label{eq:second-differential}
\end{align}
In particular, note that the common notation ``$\frac{\dd^2f}{\dx^2}$'' for the second derivative $f''(x)$ is \emph{not} justifiable in the same way that the notation $\frac{\df}{\dx}$ for the first derivative is.
That is, if we divide both sides of this equation by $\dx^2$, on the right we do \emph{not} obtain just $f''(x)$:
\[ \frac{\dd^2f}{\dx^2} = f''(x) + f'(x) \, \frac{\dd^2x}{\dx^2} \]
This is of course very similar to the problem we had with defining ``first derivatives'' of multivariable functions.
In that case we introduced the ``partial derivative'' notation, stylizing the $\dd$ into $\partial$, which indicates not an actual \emph{quotient}, but only the coefficient of a differential.
We can do the same here:
\[ \dd^2 f = \pdertwo{f}{x} \,\dx^2 + \frac{\partial^2 f}{\partial^2 x} \,\dd^2x \]
That is, we \emph{can} write the second derivative $f''(x)$ as $\pdertwo{f}{x}$.
And we can also write the first derivative $f'(x)$ as $\frac{\partial^2 f}{\partial^2 x}$ (but why would we want to?).

\begin{adv}
  A little something is being slid under the rug here.
  When we treat $\df$ as a function of two variables $x$ and ``$\dx$'' and take its differential, we ought to obtain a function of four variables: the orginal $x$ and $\dx$ and their differentials, where the new appearances of ``the differential of $x$'' are distinct from the original variable $\dx$.
  That is, there ought to be two ``$\dx$'' variables, say $\dx_1$ and $\dx_2$, and we would have
  \[ \dd^2 f = f''(x) \,\dx_1\,\dx_2 + f'(x) \,\dd(\dx_1). \]
  We are silently treating these two copies of ``$\dx$'' as the \emph{same} variable.
  This can be justified, but since second differentials are not of huge importance, we will not take the time to do so.
\end{adv}

\begin{rmk}
  We emphasize again that it is \emph{not} correct in general to write ``$\dd^2f = f''(x) \,\dx^2$''.
  This is not just for formal reasons; such an incorrect formula would lead to an incorrect ``chain rule'' for second derivatives.
  To see this, consider a composite function $y = f(g(x))$, with $u = g(x)$ the intermediate variable so that $y = f(u)$.
  Then the incorrect formula would claim that $\dd^2y = f''(u)\,\du^2 $, so that since $\du = g'(x) \,\dx$ we would have
  \[ \dd^2y = f''(u)\,(g'(x)\,\dx)^2 = f''(g(x))\, [g'(x)]^2 \,\dx^2 \qquad\text{(WRONG!)} \]
  so that the second derivative of $y$ with respect to $x$ would be $f''(g(x))\, [g'(x)]^2$.

  However, this is \emph{not} the case, as can easily be seen with an example.
  Let $f(u) = \sin u$ and $g(x) = x^2$, so that $y = (f\circ g)(x) = \sin(x^2)$.
  Then $(f\circ g)'(x) = 2x\cos (x^2)$, and thus
  \[ (f\circ g)''(x) = 2 \cos (x^2) - 4x^2 \sin(x^2) \]
  whereas
  \[ f''(g(x))\, [g'(x)]^2 = -\sin(x^2) \cdot (2x)^2 = -4x^2\sin(x^2) \]
  is only the \emph{second} term of $(f\circ g)''(x)$.

  The correct ``chain rule for second derivatives'' can be derived from \cref{eq:second-differential}:
  \begin{align*}
    \dd^2y &= f''(u) \,\du^2 + f'(u) \,\dd^2u\\
    &= f''(g(x)) \, (g'(x) \, \dx)^2 + f'(g(x)) \, \Big(g''(x) \, \dx^2 + g'(x) \, \dd^2 x\Big)\\
    &= \Big( f''(g(x)) \,[g'(x)]^2 + f'(g(x))\, g''(x)\Big) \, \dx^2 + f'(g(x))\, g'(x)\,\dd^2x.
  \end{align*}
  Thus, the second derivative is
  \[ \pdertwo{y}{x} = (f\circ g)''(x) = f''(g(x)) \,[g'(x)]^2 + f'(g(x))\, g''(x). \]
\end{rmk}

What kind of a thing is $\dd^2x$? \dots %TODO

\section{The multivariable case}
\label{sec:multivariable-second-differentials}

% TODO: an example or two.  Second partials, equality of mixed partials.  Hessian matrix and the failure of "Cauchy's invariant rule".


\end{document}
