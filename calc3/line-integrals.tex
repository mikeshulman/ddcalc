\documentclass[12pt]{amsart}
\usepackage{fullpage}
\makeatletter

%% Theorems
\usepackage{xcolor}
\definecolor{darkgreen}{rgb}{0,0.45,0} 
\usepackage[pagebackref,colorlinks,citecolor=darkgreen,linkcolor=darkgreen]{hyperref}
\usepackage{cleveref,aliascnt}
\def\defthm#1#2#3{%
  %% Ensure all theorem types are numbered with the same counter
  \newaliascnt{#1}{thm}
  \newtheorem{#1}[#1]{#2}
  \aliascntresetthe{#1}
  %% Tell cleveref's \cref what to call things
  \crefname{#1}{#2}{#3}}
\newtheorem{thm}{Theorem}[section]
\crefname{thm}{Theorem}{Theorems}
\theoremstyle{definition}
\defthm{defn}{Definition}{Definitions}
%\theoremstyle{remark}
\defthm{rmk}{Remark}{Remarks}
\defthm{advrmk}{Advanced Remark}{Advanced Remarks}
\newenvironment{adv}{\SMALL\begin{advrmk}}{\end{advrmk}}
\defthm{eg}{Example}{Examples}
\defthm{egs}{Examples}{Examples}

%% Reference format for sections
\crefformat{section}{\S#2#1#3}
\Crefformat{section}{Section~#2#1#3}
\crefrangeformat{section}{\S\S#3#1#4--#5#2#6}
\Crefrangeformat{section}{Sections~#3#1#4--#5#2#6}
\crefmultiformat{section}{\S\S#2#1#3}{ and~#2#1#3}{, #2#1#3}{ and~#2#1#3}
\Crefmultiformat{section}{Sections~#2#1#3}{ and~#2#1#3}{, #2#1#3}{ and~#2#1#3}
\crefrangemultiformat{section}{\S\S#3#1#4--#5#2#6}{ and~#3#1#4--#5#2#6}{, #3#1#4--#5#2#6}{ and~#3#1#4--#5#2#6}
\Crefrangemultiformat{section}{Sections~#3#1#4--#5#2#6}{ and~#3#1#4--#5#2#6}{, #3#1#4--#5#2#6}{ and~#3#1#4--#5#2#6}

%% Use the theorem counter for equations as well
\let\c@equation\c@thm
\numberwithin{equation}{section}

%% Pictures
\usepackage{tikz}

%% Differentials
\newcommand{\dd}{\ensuremath{\mathrm{d}}}
\newcommand{\dx}{\ensuremath{\dd x}}
\newcommand{\dy}{\ensuremath{\dd y}}
\newcommand{\dz}{\ensuremath{\dd z}}
\newcommand{\dt}{\ensuremath{\dd t}}
\newcommand{\du}{\ensuremath{\dd u}}
\newcommand{\dv}{\ensuremath{\dd v}}
\newcommand{\df}{\ensuremath{\dd f}}
\newcommand{\dg}{\ensuremath{\dd g}}
\newcommand{\dr}{\ensuremath{\dd r}}
\newcommand{\drho}{\ensuremath{\dd \rho}}
\newcommand{\dtheta}{\ensuremath{\dd \theta}}
\newcommand{\dphi}{\ensuremath{\dd \phi}}
\newcommand{\dpx}{\ensuremath{\dd \pt{x}}}

%% Partial derivatives
\newcommand{\pder}[2]{\frac{\partial #1}{\partial #2}}

%% Transposing vectors into forms
\newcommand{\tr}[1]{{#1}^\top}
\newcommand{\ptr}[1]{{(#1)}^\top}

%% Hodge star on forms
\newcommand{\str}[1]{{#1}^*}
\newcommand{\pstr}[1]{{(#1)}^*}

%% Points and vectors
\newcommand{\pt}[1]{\mathbf{#1}}
\newcommand{\ptc}[1]{(#1)}
\newcommand{\vc}[1]{\vec{#1}}
\newcommand{\vcc}[1]{(#1)}

%% Vector operations
\newcommand{\mgn}[1]{\Vert #1 \Vert}
\newcommand{\dotp}[2]{\langle #1 , #2 \rangle}
\newcommand{\crossp}[2]{#1 \times #2}

%% Fractions
\newcommand{\half}{\ensuremath{\textstyle\frac{1}{2}}}
\newcommand{\third}{\ensuremath{\textstyle\frac{1}{3}}}

%% Orders
\renewcommand{\th}{^{\mathrm{th}}}
\newcommand{\st}{^{\mathrm{st}}}

%% Approximation to specified order
\newcommand{\apx}[1]{\mathrel{\overset{#1}{\approx}}}

\makeatother

\title{Line integrals}
\begin{document}
\maketitle

We introduced differential forms by saying that when you wrote
\[ \int_a^b f(x) \,\dx \]
in one-variable calculus, you were actually integrating not the function $f$ but the differential form $f(x)\,\dx$.
We will now make precise what it means to integrate a differential form, simultaneously generalizing to the multivariable situation.

\section{Integrating forms in one dimension}
\label{sec:integrating-forms}

When we ``integrate a function $f$'' in one-variable calculus from $a$ to $b$, we divide up the interval $[a,b]$ into a large number $N$ of small subintervals $[x_i,x_{i+1}]$ of width $\Delta x_i$, and consider the \emph{Riemann sum}
\[ \sum_{i=1}^N f(\rstag i)\, \Delta x_i \]
where $\rstag i$ is some point in $[x_i,x_{i+1}]$.
Then we take a limit as the subinterval widths $\Delta x_i$ go to zero.

If we regard this instead as integrating the differential form $\omega(x,\dx) = f(x)\, \dx$, we can rewrite the Riemann sum as
\[ \sum_{i=1}^N \omega(\rstag i,\Delta x_i). \]
In other words, the differential form specifies, for each value $\rstag i$ of $x$ and each subinterval width $\Delta x_i$, the appropriate (approximate) contribution $\omega(\rstag i,\Delta x_i)$ to the integral.
We simply add up these contributions, and then take the limit.

We can now generalizes this to integrate \emph{any} differential form in one dimension.
We simply define
\[ \int_a^b \omega = \lim_{\Delta x_i \to 0} \sum_{i=1}^N \omega(\rstag i,\Delta x_i). \]
if this limit exists.
Note that when integrating a differential form, we do not write a ``$\dx$'' at the end of the integral: we simply write $\int_a^b \omega$.
The $\dx$ at the end of an ordinary integral is \emph{part} of the differential form being integrated.

\begin{eg}
  As a first new example, consider the form $f(x) \, |\dx|$, where $f$ is some continuous function.
  If $a<b$, so that each $\Delta x_i$ is positive, then the Riemann sums for $\int_a^b f(x) \, |\dx|$ are the same as those for $\int_a^b f(x) \, \dx$, and thus the integral is the same.
  The difference is only visible if we consider also the case when $b<a$, so that each $\Delta x_i$ is negative.
  In one-variable calculus, you probably learned that
  \[ \int_b^a f(x)\,\dx = - \int_a^b f(x) \,\dx. \]
  This is true because the Riemann sums for the first integral are the negatives of those for the second.
  However, because of the absolute value, the Riemann sums for $\int_a^b f(x) \, |\dx|$ are the \emph{same} as those for $\int_b^a f(x) \, |\dx|$; thus we have
  \[ \int_b^a f(x)\,|\dx| = \int_a^b f(x) \,|\dx|. \]
\end{eg}

\begin{eg}
  Consider the form $\dx^2$.
  If we have a tagged partition with $\Delta x_i < \epsilon$ for all $i$, then the corresponding Riemann sum is
  \begin{align*}
    \sum_{i=1}^N (\Delta x_i)^2 &< \epsilon \sum_{i=1}^N (\Delta x_i)\\
    &= \epsilon (b-a).
  \end{align*}
  In the limit, $\epsilon \to 0$, and thus the Riemann sums also go to zero.
  So we have
  \[ \int_a^b \dx^2 = 0. \]
  Intuitively, we may say that an ordinary integral $\int_a^b f(x) \,\dx$ obtains a zeroth-order result by adding up a large number of small first-order changes; but if we try to add up the \emph{same} number of \emph{second-order} changes, the result will be negligible.
\end{eg}

More generally, we have the following:

\begin{thm}\label{thm:int-gtfirstorder-onevar}
  If $\omega$ is greater than first order, then
  \[ \int_a^b \omega \;=\; 0\]
  for any closed interval $[a,b]$.
\end{thm}

\begin{adv}
  The full generality of \cref{thm:int-gtfirstorder-onevar} actually requires a slightly more powerful definition of integration than is usually taught in one-variable calculus, called ``Henstock integration''.
  Roughly, we need to be more flexible with the exact meaning of what it means for the $\Delta x_i$ to all approach zero at once.
  Since we will not be making this very precise anyway, we henceforth ignore this issue.
\end{adv}

\section{Line integrals}
\label{sec:line-integrals}

If we want to integrate a form in more than one dimension, we have a new problem: we need more information to specify the ``interval'' over which to integrate.


\end{document}
