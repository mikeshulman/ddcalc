\documentclass[12pt]{amsart}
\usepackage{fullpage}
\makeatletter

%% AMS
\usepackage{amsmath,amssymb,latexsym}

%% Theorems
\usepackage{xcolor}
\definecolor{darkgreen}{rgb}{0,0.45,0} 
\usepackage[pagebackref,colorlinks,citecolor=darkgreen,linkcolor=darkgreen]{hyperref}
\usepackage{cleveref,aliascnt}
\usepackage{amsthm}
\def\defthm#1#2#3{%
  %% Ensure all theorem types are numbered with the same counter
  \newaliascnt{#1}{thm}
  \newtheorem{#1}[#1]{#2}
  \aliascntresetthe{#1}
  %% Tell cleveref's \cref what to call things
  \crefname{#1}{#2}{#3}}
\newtheorem{thm}{Theorem}[section]
\crefname{thm}{Theorem}{Theorems}
\theoremstyle{definition}
\defthm{defn}{Definition}{Definitions}
%\theoremstyle{remark}
\defthm{rmk}{Remark}{Remarks}
\defthm{advrmk}{Advanced Remark}{Advanced Remarks}
\newenvironment{adv}{\small\begin{advrmk}}{\end{advrmk}}
\defthm{eg}{Example}{Examples}
\defthm{egs}{Examples}{Examples}
\defthm{sage}{Sage}{Sages}

%% Reference format for sections
\crefformat{section}{\S#2#1#3}
\Crefformat{section}{Section~#2#1#3}
\crefrangeformat{section}{\S\S#3#1#4--#5#2#6}
\Crefrangeformat{section}{Sections~#3#1#4--#5#2#6}
\crefmultiformat{section}{\S\S#2#1#3}{ and~#2#1#3}{, #2#1#3}{ and~#2#1#3}
\Crefmultiformat{section}{Sections~#2#1#3}{ and~#2#1#3}{, #2#1#3}{ and~#2#1#3}
\crefrangemultiformat{section}{\S\S#3#1#4--#5#2#6}{ and~#3#1#4--#5#2#6}{, #3#1#4--#5#2#6}{ and~#3#1#4--#5#2#6}
\Crefrangemultiformat{section}{Sections~#3#1#4--#5#2#6}{ and~#3#1#4--#5#2#6}{, #3#1#4--#5#2#6}{ and~#3#1#4--#5#2#6}

%% Use the theorem counter for equations as well
\let\c@equation\c@thm
\numberwithin{equation}{section}

%% Pictures
\usepackage{tikz}
\usepackage{graphicx}

%% Differentials
\newcommand{\dd}{\ensuremath{\mathrm{d}}}
\newcommand{\ed}{\ensuremath{\mathrm{d}\wedge}}

%% Differentials of variables
\newcommand{\dx}{{\ensuremath{\dd x}}}
\newcommand{\dy}{{\ensuremath{\dd y}}}
\newcommand{\dz}{{\ensuremath{\dd z}}}
\newcommand{\dzbar}{{\ensuremath{\dd \bar{z}}}}
\newcommand{\dt}{{\ensuremath{\dd t}}}
\newcommand{\du}{{\ensuremath{\dd u}}}
\newcommand{\dv}{{\ensuremath{\dd v}}}
\newcommand{\df}{{\ensuremath{\dd f}}}
\newcommand{\dg}{{\ensuremath{\dd g}}}
\renewcommand{\dh}{{\ensuremath{\dd h}}}
\newcommand{\dr}{{\ensuremath{\dd r}}}
\newcommand{\drho}{{\ensuremath{\dd \rho}}}
\newcommand{\dtheta}{{\ensuremath{\dd \theta}}}
\newcommand{\dphi}{{\ensuremath{\dd \phi}}}
\newcommand{\dpx}{{\ensuremath{\dd \pt{x}}}}

%% Not differentials of anything
\usepackage[T1]{fontenc}
\usepackage{lmodern}
\newcommand{\dl}{\ensuremath{\mathrm{\dj}\ell}}
\newcommand{\dA}{\ensuremath{\mathrm{\dj}A}}
\newcommand{\dV}{\ensuremath{\mathrm{\dj}V}}
\newcommand{\dn}{\ensuremath{\mathrm{\dj}\vc{n}}}

%% Wedge product
%\newcommand{\wedge}{\wedge}

%% Partial derivatives
% \newcommand{\pder}[2]{\frac{\partial #1}{\partial #2}}
% \newcommand{\pdertwo}[2]{\frac{\partial^2 #1}{\partial #2^2}}
% \newcommand{\pdermixed}[3]{\frac{\partial^2 #1}{\partial #2 \,\partial #3}}
\newcommand{\pder}[2]{{#1}_{#2}}
\newcommand{\pdertwo}[2]{{#1}_{{#2}{#2}}}
\newcommand{\pdermixed}[3]{{#1}_{{#2}{#3}}}

%% Transposing vectors into forms
\newcommand{\tr}[1]{{#1}^\top}
\newcommand{\ptr}[1]{{(#1)}^\top}

%% Hodge star on forms
\newcommand{\str}[1]{{#1}^*}
\newcommand{\pstr}[1]{{(#1)}^*}

%% Points and vectors
\newcommand{\pt}[1]{\mathbf{#1}}
\newcommand{\ptc}[1]{(#1)}
\newcommand{\ptx}{\pt{x}}
% \newcommand{\vc}[1]{\vec{#1}}
\newcommand{\vc}[1]{\mathbf{#1}}
\newcommand{\vcc}[1]{(#1)}
\newcommand{\lrvcc}[1]{\left(#1\right)}

%% Vector operations
\newcommand{\mgn}[1]{\Vert #1 \Vert}
%\newcommand{\dotp}[2]{\langle #1 , #2 \rangle}
\newcommand{\dotp}[2]{#1 \cdot #2}
\newcommand{\crossp}[2]{#1 \times #2}

%% Vector calculus operations
\newcommand{\grad}[1]{\nabla #1}
\renewcommand{\div}[1]{\nabla \cdot #1}
\newcommand{\curl}[1]{\nabla \times #1}

%% Fractions
\newcommand{\half}{\ensuremath{\textstyle\frac{1}{2}}}
\newcommand{\halfi}{\ensuremath{\textstyle\frac{1}{2i}}}
\newcommand{\halfpii}{\ensuremath{\textstyle\frac{1}{2\pi i}}}
\newcommand{\third}{\ensuremath{\textstyle\frac{1}{3}}}

%% Orders
\renewcommand{\th}{^{\mathrm{th}}}
\newcommand{\st}{^{\mathrm{st}}}

%% Approximation to specified order
\newcommand{\apx}[1]{\mathrel{\overset{#1}{\approx}}}

%% Riemann sums
\newcommand{\rstag}[2]{{#1}_{#2}^*}

%% Complex conjugation
\newcommand{\conj}[1]{\bar{#1}}
\newcommand{\zbar}{\conj{z}}
\newcommand{\wbar}{\conj{w}}

%% Parametrized curves
\newcommand{\curve}{C}
\newcommand{\curvept}{\mathbf{x}}
\newcommand{\dcurvept}{\dd\mathbf{x}}
\newcommand{\curvex}{x}
\newcommand{\curvey}{y}
\newcommand{\curvez}{z}

%% Regions
\newcommand{\region}{R}
\newcommand{\bdry}{\partial}

%% Parametrized surfaces
\newcommand{\surf}{S}
\newcommand{\surfx}{x}
\newcommand{\surfy}{y}
\newcommand{\surfz}{z}

%% Integrals of differential forms.  The only argument is the
%% integration manifold; the differential form just comes afterwards.
\newcommand{\lint}[1]{\int_{#1}} % line integral over a curve
\newcommand{\sint}[1]{\iint_{#1}} % surface integral form over a surface
\newcommand{\vint}[1]{\iiint_{#1}} % volume integral over a spatial region

%% "Classical" integrals.  For these the integrand is a *function*
%% (well, technically, an expression with free variables) which is
%% given as a second argument.
\newcommand{\dint}[2]{\iint_{#1} #2 \,\dA} % double integral over a plane region
\newcommand{\tint}[2]{\iiint_{#1} #2 \,\dV} % triple integral over a spatial region
\newcommand{\itint}[4]{\int_{#2}^{#3} #4 \,\dd #1} % ordinary one-variable integral over an interval
\newcommand{\itdint}[7]{\int_{#2}^{#3} \int_{#5}^{#6} #7 \,\dd #4\,\dd #1} % double iterated integral
% Can't have more than 9 parameters
%\newcommand{\ittint}[10]{\int_{#2}^{#3} \int_{#5}^{#6} \int_{#8}^{#9} #10 \,\dd #7\,\dd #4\,\dd #1} % triple iterated integral
\makeatletter
\newcommand{\ittint}[3]{\int_{#2}^{#3} \@ittint #1}
\newcommand{\@ittint}[8]{\int_{#3}^{#4} \int_{#6}^{#7} #8 \,\dd #5\,\dd #2\,\dd #1} % triple iterated integral
\makeatother

%% Textbooks
\usepackage{version}
\includeversion{notextbook}
\excludeversion{stewart}
\excludeversion{hugheshallett}
\excludeversion{rogawski}

\makeatother

\title{Line integrals}
\begin{document}
\maketitle

We introduced differential forms by saying that when you wrote
\[ \int_a^b f(x) \,\dx \]
in one-variable calculus, you were actually integrating not the function $f$ but the differential form $f(x)\,\dx$.
We will now make precise what it means to integrate a differential form, simultaneously generalizing to the multivariable situation.

\section{Integrating forms in one dimension}
\label{sec:integrating-forms}

When we ``integrate a function $f$'' in one-variable calculus from $a$ to $b$, we divide up the interval $[a,b]$ into a large number $N$ of small subintervals $[x_{i-1},x_i]$ of width $\Delta x_i$, with $x_0 = a$ and $x_N=b$, and consider the \emph{Riemann sum}
\[ \sum_{i=1}^N f(\rstag x i)\, \Delta x_i \]
where $\rstag x i$ is some point in $[x_{i-1},x_i]$.
Then we take a limit as the subinterval widths $\Delta x_i$ go to zero.

If we regard this instead as integrating the differential form $\omega(x,\dx) = f(x)\, \dx$, we can rewrite the Riemann sum as
\[ \sum_{i=1}^N \omega(\rstag x i,\Delta x_i). \]
In other words, the differential form specifies, for each value $\rstag x i$ of $x$ and each subinterval width $\Delta x_i$, the appropriate (approximate) contribution $\omega(\rstag x i,\Delta x_i)$ to the integral.
We simply add up these contributions, and then take the limit.

We can now generalizes this to integrate \emph{any} differential form in one dimension.
We simply define
\[ \int_a^b \omega = \lim_{\Delta x_i \to 0} \sum_{i=1}^N \omega(\rstag x i,\Delta x_i). \]
if this limit exists.
Note that when integrating a differential form, we do not write a ``$\dx$'' at the end of the integral: we simply write $\int_a^b \omega$.
The $\dx$ at the end of an ordinary integral is \emph{part} of the differential form being integrated.

\begin{eg}\label{eg:onevar-absint}
  As a first new example, consider the form $f(x) \, |\dx|$, where $f$ is some continuous function.
  If $a<b$, so that each $\Delta x_i$ is positive, then the Riemann sums for $\int_a^b f(x) \, |\dx|$ are the same as those for $\int_a^b f(x) \, \dx$, and thus the integral is the same.
  The difference is only visible if we consider also the case when $b<a$, so that each $\Delta x_i$ is negative.
  In one-variable calculus, you probably learned that
  \[ \int_b^a f(x)\,\dx = - \int_a^b f(x) \,\dx. \]
  This is true because the Riemann sums for the first integral are the negatives of those for the second.
  However, because of the absolute value, the Riemann sums for $\int_a^b f(x) \, |\dx|$ are the \emph{same} as those for $\int_b^a f(x) \, |\dx|$; thus we have
  \[ \int_b^a f(x)\,|\dx| = \int_a^b f(x) \,|\dx|. \]
  In conclusion, therefore, we can say that
  \begin{equation}
    \int_a^b f(x) \,|\dx| =
    \begin{cases}
      \displaystyle\int_a^b f(x) \,\dx \qquad\text{if } a<b\\[10pt]
      \displaystyle\int_b^a f(x) \,\dx \qquad\text{if } b<a.
    \end{cases}\label{eq:onevar-absint}
  \end{equation}
  In other words, the differential form $f(x)\,\dx$ notices the ``direction'' (or \emph{orientation}) of the interval over which it is integrated, and gives a different result whether we integrate it ``from $a$ to $b$'' or ``from $b$ to $a$'' --- whereas the form $f(x)\,|\dx|$ does not notice the direction.
\end{eg}

\begin{eg}
  Consider the form $\dx^2$.
  If we have a division into subintervals $[x_{i-1},x_i]$ such that $\Delta x_i < \varepsilon$ for all $i$, then the corresponding Riemann sum is
  \begin{align*}
    \sum_{i=1}^N (\Delta x_i)^2 &< \varepsilon \sum_{i=1}^N (\Delta x_i)\\
    &= \varepsilon (b-a).
  \end{align*}
  In the limit, $\varepsilon \to 0$, and thus the Riemann sums also go to zero.
  So we have
  \[ \int_a^b \dx^2 = 0. \]
  Intuitively, we may say that an ordinary integral $\int_a^b f(x) \,\dx$ obtains a zeroth-order result by adding up a large number of small first-order changes; but if we try to add up the \emph{same} number of \emph{second-order} changes, the result will be negligible.
\end{eg}

More generally, we have the following:

\begin{thm}\label{thm:int-negligible-onevar}
  If $\omega$ is negligible, then for any closed interval $[a,b]$ we have
  \[ \int_a^b \omega \;=\; 0.\]
\end{thm}

\begin{adv}
  The full generality of \cref{thm:int-negligible-onevar} actually requires a slightly more powerful definition of integration than is usually taught in one-variable calculus, called ``Henstock integration''.
  Roughly, we need to be more flexible with the exact meaning of what it means for the $\Delta x_i$ to all approach zero at once.
  Since we will not be making this very precise anyway, we henceforth ignore this issue.
\end{adv}

In particular, it follows that
\begin{equation}\label{eq:int-apx1-equal}
  \text{If }\omega\apx1 \eta, \text{then} \displaystyle\int_a^b \omega = \int_a^b \eta.
\end{equation}
For example, recall that if $f$ is differentiable, then
\begin{align*}
  f(x+\dx)-f(x) &\apx1 \df. %\\
  % &= f'(x) \,\dx
\end{align*}
Therefore, we have
\[ \int_a^b \df = \int_a^b (f(x+\dx) - f(x)). \]
Now note that for any partition of $[a,b]$ into subintervals, we have $x_{i-1} + \dx = x_i$, by definition.
Thus, if we choose the left-hand endpoints (that is, $\rstag x i = x_{i-1}$), then the Riemann sum for $\int_a^b (f(x+\dx) - f(x))$ will be
\begin{align*}
  \Big(f(x_1) - f(x_0)\Big) + \Big(f(x_2)-f(x_1)\Big) + \cdots + \Big(f(x_N) - f(x_{N-1})\Big)
  &= f(x_N) - f(x_0)\\
  &= f(b) - f(a).
\end{align*}
Therefore,
\begin{equation}
  \int_a^b \df = f(b) - f(a).\label{eq:differential-ftc-eval}
\end{equation}
Recalling that $\df = f'(x)\,\dx$, we see that this is half of the Fundamental Theorem of Calculus.
\begin{stewart}(The textbook calls this half the ``Evaluation Theorem''.)\end{stewart}

The other half of the Fundamental Theorem of Calculus says that every continuous function has an antiderivative: i.e.\ if $f$ is continuous, then there is a function $F$ such that $F'(x) = f(x)$.
This is equivalent to saying that any \emph{linear} differential form $f(x)\,\dx$ is the differential $\dd F = F'(x)\,\dx$ of some function.
(It follows from this and \cref{eq:differential-ftc-eval} that one such function $F$ is $F(x)= \int_a^x f(x)\,\dx$, and this is often included as part of the Fundamental Theorem.)
Therefore, at least in principle, we can evaluate the integral of \emph{any} linear differential form using \cref{eq:differential-ftc-eval}.
(Although, as you probably learned in your previous calculus classes, in practice there are many functions whose antiderivatives have no simple formula.)

This is not the case for nonlinear differential forms.
Forms such as $|\dx|$ and $\dx^2$ and $e^\dx$ are not the differential of any function, so we cannot evaluate their integrals using \cref{eq:differential-ftc-eval}.
Of the nonlinear forms, the ones which arise most in practice are those of the form $f(x)\,|\dx|$, and those we can integrate using \cref{eq:onevar-absint}.
In many other cases, we can apply \cref{thm:int-negligible-onevar} or \cref{eq:int-apx1-equal}.


\section{Line integrals}
\label{sec:line-integrals}

If we want to integrate a form in more than one dimension, we have a new problem: we need more information to specify the ``interval'' over which to integrate.
The most appropriate sort of ``interval'' turns out to be a \textbf{parametrized curve} (or \emph{parametric curve}), a notion which you might have encountered in one-variable calculus.

Recall that a parametrized curve $\curve$ in two dimensions consists of a \emph{parameter interval} $[a,b]$ and two functions $\curvex(t)$ and $\curvey(t)$ defined for all $t\in [a,b]$.
The curve is then the set of points $\curvept(t) = (\curvex(t),\curvey(t))$ as $t$ ranges over some domain (we call $t$ the \emph{parameter}).
Similarly, in three dimensions we have three functions $\curvex(t)$, $\curvey(t)$, and $\curvez(t)$ and we consider the points $\curvept(t) = (\curvex(t),\curvey(t),\curvez(t))$.
We say that a parametrized curve is \textbf{continuous} or \textbf{differentiable} if the functions $\curvex$, $\curvey$, and (possibly) $\curvez$ are continuous or differentiable.

Now suppose $\omega$ is a differential form in two dimensions, which we want to integrate over a parametrized curve $\curve$ defined on $[a,b]$.
We regard $\omega(\ptx,\dpx)$ as telling us about the first-order changes in some quantity Q as $\ptx$ changes to $\ptx+\dpx$, whereas we are interested in the total change of Q ``along $\curve$''.
We can approximate this total change by dividing $[a,b]$ into a large number $N$ of subintervals $[t_{i-1},t_i]$, and choosing a point $\rstag t i$ in each such subinterval.
Then if we consider the part of $\curve$ between $t_{i-1}$ and $t_i$ to be approximately a straight line, we can approximate the first-order change in Q along that part by $\omega(\curvept(\rstag t i),\Delta \curvept_i)$, where $\Delta\curvept_i = \curvept(t_i)-\curvept(t_{i-1})$.
Adding these up, we obtain the Riemann sum
\[ \sum_{i=1}^N \omega(\curvept(\rstag t i),\Delta \curvept_i) \]
and we define the \textbf{line integral} of $\omega$ over $\curve$ to be the limit of these sums, as the widths $\Delta t_i = t_i - t_{i-1}$ go to zero:
\begin{equation}
  \lint{\curve} \omega = \lim_{\Delta t_i \to 0} \sum_{i=1}^N \omega(\curvept(\rstag t i),\Delta \curvept_i).\label{eq:line-integral}
\end{equation}
(It would be more appropriate to call this a \textit{curve integral}, since $\curve$ is a curve and not usually a straight line, but the term ``line integral'' is standard.)

If the curve $\curve$ is differentiable, then we can write
\[\dx = x'(t)\,\dt \qquad \dy = y'(t) \,\dt \qquad \dz = z'(t)\,\dt \]
and we might expect to be able to evaluate a line integral by substituting these values to get an ordinary integral in terms of $t$.
This is in fact possible, although it is not trivial to prove; it is a special case of the \emph{Change of Variables Theorem}.
Here we have another situation like the chain rule, where the notation of differentials makes a deep fact appear obvious and easy-to-use.

TODO: Prove the CoVT for this definition of the line integral.  Note that it's subtly different from the usual definition, where we would evaluate $\omega$ on $\ptx'(\rstag t i)$ rather than $\Delta\curvept_i$.

\begin{eg}
  Suppose $x(t) = t^2$ and $y(t) = t^3$ define a curve in two dimensions with domain $[0,2]$, so that $\dx = 2t\,\dt$ and $\dy = 3t^2\,\dt$.
  Then we can evaluate a line integral such as
  \[ \lint{\curve} (x^2 y \,\dx - x \,\dy) \]
  by substituting these values for $x$, $y$, $\dx$, and $\dy$, obtaining the ordinary integral
  \[ \int_{t=0}^2 \Big( (t^2)^2(t^3)(2t\,\dt) - (t^2)(3t^2\,\dt)\Big)
  = \int_{t=0}^2 (2t^8 - 3t^4)\,\dt\]
  which we can evaluate using the ordinary fundamental theorem of calculus.
\end{eg}

Recall that differential forms are functions whose input is a point and a vector.
Thus, another way to state the Change of Variables Theorem for line integrals is that
\begin{equation}
  \lint{\curve} \omega = \int_a^b \omega\big(\curvept(t),\, \curvept'(t) \,\dt\big)\label{eq:lineint-chvar-2}
\end{equation}
where $\curvept'(t)$ denotes the vector $(\curvex'(t),\curvey'(t))$ or $(\curvex'(t),\curvey'(t),\curvez'(t))$.
If $\omega$ is a \emph{linear} differential form (recall that means it looks like $f(\ptx)\,\dx + g(\pt{y})\,\dy + h(\pt{z})\,\dz$), then
\[ \omega\big(\curvept(t),\, \curvept'(t) \,\dt\big) = \omega\big(\curvept(t),\, \curvept'(t)\big)\,\dt \]
and thus the right-hand side of \cref{eq:lineint-chvar-2} becomes
\[ \int_a^b \omega\big(\curvept(t),\, \curvept'(t)\big)\,\dt. \]
This is an ordinary one-variable integral of the sort we already understand.
Therefore, the line integral of any linear differential form over any differentiable curve can be reduced to a one-variable integral, which we can at least hope to integrate using the fundamental theorem of calculus.
More generally, if $\curve$ is only \emph{piecewise} differentiable, then we can break a line integral up into several integrals and apply this method on each piece.
\begin{notextbook}Your textbook probably contains more details.\end{notextbook}
\begin{stewart}More details are in the textbook (section 13.2).\end{stewart}

\begin{rmk}\label{rmk:lineint-orientation}
  Although we used a particular parametrization of the curve $\curve$ to define the line integral $\lint{\curve} \omega$ in \cref{eq:line-integral}, and to evaluate it as in \cref{eq:lineint-chvar-2}, in fact the value of the integral is independent of the parametrization.
  This makes sense if we look at the definition of the Riemann sums in \cref{eq:line-integral}, which depend only on the values of $\omega$ at the points $\curvept(\rstag t i)$ and the vectors $\Delta \curvept_i$; any other parametrization would induce a corresponding division into subintervals with the same points and vectors.
  
  TODO: We also need the limits as $\Delta t_i\to 0$ to be the same limit.  Does this require the parametrizations to be differentiably (or perhaps Lipschitz-ly) related?

  The integral does, however, depend on the \emph{orientation} of the curve $\curve$, i.e.\ the direction in which the parametrization traverses it.
  Reversing the orientation would negate the vectors $\Delta \curvept_i$, and for a \emph{linear} differential form, this would negate the value of the integral.
  (In the next section, we will consider line integrals that are independent of orientation.)
\end{rmk}

Finally, recall that any vector field $\vc{v}$ gives rise to a linear differential form $\dotp{\vc{v}(\ptx)}{\dpx}$.
We can then integrate this form over any curve, obtaining a line integral denoted
\[ \lint{\curve} \dotp{\vc{v}(\ptx)}{\dpx} \]
or any number of similar notations.
\begin{notextbook}You will probably find this in your textbook.\end{notextbook}
\begin{stewart}The textbook writes $\lint{\curve} \dotp{\vc{v}}{\dd\pt{r}}$ (section 13.2).\end{stewart}


\section{Arc length}
\label{sec:arc-length}

The most important example of a nonlinear line integral is one that you may already have encountered in one-variable calculus: the \emph{arc length} of a parametrized curve.
To find this as an integral, we should choose $\omega$ to be a first-order approximation to the length of a bit of curve.
If we consider the curve to be approximately a line, then this would be just the length of the vector $\dpx$.
Thus, in two dimensions we should integrate the differential form
\[ \sqrt{\dx^2+\dy^2}\]
while in three dimensions we should integrate
\[ \sqrt{\dx^2+\dy^2 + \dz^2}.\]
It is traditional to denote these differential forms by $\mathrm{d}\ell$ or $\mathrm{d}s$, but this is potentially confusing because they are not the differential of anything.
\begin{stewart}(The textbook uses $ds$.)\end{stewart}
We will instead denote them by $\dl$ (note the stroke over the $\mathrm{d}$).
That is, we define
\[ \dl = \mgn{\dpx} =
\begin{cases}
  \sqrt{\dx^2+\dy^2}\qquad\text{in two dimensions}\\[3pt]
  \sqrt{\dx^2+\dy^2+\dz^2}\qquad\text{in three dimensions}.
\end{cases}\]
The form $\dl$ is sometimes called the \textbf{length element} or the \textbf{line element}.

Now the \textbf{arc length} of a curve $\curve$ is defined to be
\[ \text{arc length of } \curve = \lint{\curve} \dl \]
if this integral exists.
More generally, if $f$ is a multivariable function, then the \textbf{line integral of $f$ with respect to arc length} over $\curve$ is defined to be
\[ \lint{\curve} f(\ptx)\,\dl. \]

Even though $\dl$ is not a linear differential form, the Change of Variables Theorem still applies to arc length integrals, and can reduce them to one-variable integrals.
Essentially, this is because if $\dx = x'(t) \,\dt$ and $\dy = y'(t)\,\dt$, we can write
\begin{align*}
  \dl &= \sqrt{\dx^2+\dy^2}\\
  &= \sqrt{(x'(t)\,\dt)^2 + (y'(t)\,\dt)^2}\\
  &= \sqrt{(x'(t)^2 + y'(t)^2) \,\dt^2}\\
  &= \sqrt{x'(t)^2 + y'(t)^2} \;|\dt|.
\end{align*}
and we know how to evaluate integrals of forms such as $f(t)\,|\dt|$: we use \cref{eq:onevar-absint}.
In fact, since we nearly always parametrize curves over intervals $[a,b]$ with $a<b$, the arc length integral of a function $f$ generally becomes simply
\[ \int_a^b f(x(t),y(t)) \sqrt{x'(t)^2 + y'(t)^2} \,\dt \]
in two dimensions, or
\[ \int_a^b f(x(t),y(t),z(t)) \sqrt{x'(t)^2 + y'(t)^2+z'(t)^2} \,\dt \]
in three.

%% TODO: Add an example.

\begin{rmk}
  The presence of $|\dt|$ rather than $\dt$ does imply that arc length integrals are independent not only of parametrization, but of the orientation of the curve $\curve$ (see \cref{rmk:lineint-orientation}).
  This makes sense: the length of a curve should not depend on the direction in which we traverse it.
\end{rmk}

\begin{rmk}
  We have defined $\dl$ only in two and three dimensions, but there is an analogous definition in any number of dimensions.
  In \emph{one} dimension, the definition would be
  \[ \dl = \sqrt{\dx^2} = |\dx|. \]
  Thus, the integrals considered in \cref{eg:onevar-absint} are a one-variable version of ``line integrals with respect to arc length''.
\end{rmk}

\begin{notextbook}Your textbook probably discusses arc length integrals and line integrals of vector fields.\end{notextbook}
\begin{stewart}The textbook discusses line integrals in section 11.5, starting with arc length integrals and then considering also line integrals of linear differential forms (but not by that name) and of vector fields.\end{stewart}
Our terminology of differential forms enables us to see all of these as particular cases of the same notion of integral.

\section{The fundamental theorem of calculus}
\label{sec:lineintegral-ftc}



\end{document}
