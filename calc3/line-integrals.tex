\documentclass[12pt]{amsart}
\usepackage{fullpage}
\makeatletter

%% AMS
\usepackage{amsmath,amssymb,latexsym}

%% Theorems
\usepackage{xcolor}
\definecolor{darkgreen}{rgb}{0,0.45,0} 
\usepackage[pagebackref,colorlinks,citecolor=darkgreen,linkcolor=darkgreen]{hyperref}
\usepackage{cleveref,aliascnt}
\usepackage{amsthm}
\def\defthm#1#2#3{%
  %% Ensure all theorem types are numbered with the same counter
  \newaliascnt{#1}{thm}
  \newtheorem{#1}[#1]{#2}
  \aliascntresetthe{#1}
  %% Tell cleveref's \cref what to call things
  \crefname{#1}{#2}{#3}}
\newtheorem{thm}{Theorem}[section]
\crefname{thm}{Theorem}{Theorems}
\theoremstyle{definition}
\defthm{defn}{Definition}{Definitions}
%\theoremstyle{remark}
\defthm{rmk}{Remark}{Remarks}
\defthm{advrmk}{Advanced Remark}{Advanced Remarks}
\newenvironment{adv}{\small\begin{advrmk}}{\end{advrmk}}
\defthm{eg}{Example}{Examples}
\defthm{egs}{Examples}{Examples}
\defthm{sage}{Sage}{Sages}

%% Reference format for sections
\crefformat{section}{\S#2#1#3}
\Crefformat{section}{Section~#2#1#3}
\crefrangeformat{section}{\S\S#3#1#4--#5#2#6}
\Crefrangeformat{section}{Sections~#3#1#4--#5#2#6}
\crefmultiformat{section}{\S\S#2#1#3}{ and~#2#1#3}{, #2#1#3}{ and~#2#1#3}
\Crefmultiformat{section}{Sections~#2#1#3}{ and~#2#1#3}{, #2#1#3}{ and~#2#1#3}
\crefrangemultiformat{section}{\S\S#3#1#4--#5#2#6}{ and~#3#1#4--#5#2#6}{, #3#1#4--#5#2#6}{ and~#3#1#4--#5#2#6}
\Crefrangemultiformat{section}{Sections~#3#1#4--#5#2#6}{ and~#3#1#4--#5#2#6}{, #3#1#4--#5#2#6}{ and~#3#1#4--#5#2#6}

%% Use the theorem counter for equations as well
\let\c@equation\c@thm
\numberwithin{equation}{section}

%% Pictures
\usepackage{tikz}
\usepackage{graphicx}

%% Differentials
\newcommand{\dd}{\ensuremath{\mathrm{d}}}
\newcommand{\ed}{\ensuremath{\mathrm{d}\wedge}}

%% Differentials of variables
\newcommand{\dx}{{\ensuremath{\dd x}}}
\newcommand{\dy}{{\ensuremath{\dd y}}}
\newcommand{\dz}{{\ensuremath{\dd z}}}
\newcommand{\dzbar}{{\ensuremath{\dd \bar{z}}}}
\newcommand{\dt}{{\ensuremath{\dd t}}}
\newcommand{\du}{{\ensuremath{\dd u}}}
\newcommand{\dv}{{\ensuremath{\dd v}}}
\newcommand{\df}{{\ensuremath{\dd f}}}
\newcommand{\dg}{{\ensuremath{\dd g}}}
\renewcommand{\dh}{{\ensuremath{\dd h}}}
\newcommand{\dr}{{\ensuremath{\dd r}}}
\newcommand{\drho}{{\ensuremath{\dd \rho}}}
\newcommand{\dtheta}{{\ensuremath{\dd \theta}}}
\newcommand{\dphi}{{\ensuremath{\dd \phi}}}
\newcommand{\dpx}{{\ensuremath{\dd \pt{x}}}}

%% Not differentials of anything
\usepackage[T1]{fontenc}
\usepackage{lmodern}
\newcommand{\dl}{\ensuremath{\mathrm{\dj}\ell}}
\newcommand{\dA}{\ensuremath{\mathrm{\dj}A}}
\newcommand{\dV}{\ensuremath{\mathrm{\dj}V}}
\newcommand{\dn}{\ensuremath{\mathrm{\dj}\vc{n}}}

%% Wedge product
%\newcommand{\wedge}{\wedge}

%% Partial derivatives
% \newcommand{\pder}[2]{\frac{\partial #1}{\partial #2}}
% \newcommand{\pdertwo}[2]{\frac{\partial^2 #1}{\partial #2^2}}
% \newcommand{\pdermixed}[3]{\frac{\partial^2 #1}{\partial #2 \,\partial #3}}
\newcommand{\pder}[2]{{#1}_{#2}}
\newcommand{\pdertwo}[2]{{#1}_{{#2}{#2}}}
\newcommand{\pdermixed}[3]{{#1}_{{#2}{#3}}}

%% Transposing vectors into forms
\newcommand{\tr}[1]{{#1}^\top}
\newcommand{\ptr}[1]{{(#1)}^\top}

%% Hodge star on forms
\newcommand{\str}[1]{{#1}^*}
\newcommand{\pstr}[1]{{(#1)}^*}

%% Points and vectors
\newcommand{\pt}[1]{\mathbf{#1}}
\newcommand{\ptc}[1]{(#1)}
\newcommand{\ptx}{\pt{x}}
% \newcommand{\vc}[1]{\vec{#1}}
\newcommand{\vc}[1]{\mathbf{#1}}
\newcommand{\vcc}[1]{(#1)}
\newcommand{\lrvcc}[1]{\left(#1\right)}

%% Vector operations
\newcommand{\mgn}[1]{\Vert #1 \Vert}
%\newcommand{\dotp}[2]{\langle #1 , #2 \rangle}
\newcommand{\dotp}[2]{#1 \cdot #2}
\newcommand{\crossp}[2]{#1 \times #2}

%% Vector calculus operations
\newcommand{\grad}[1]{\nabla #1}
\renewcommand{\div}[1]{\nabla \cdot #1}
\newcommand{\curl}[1]{\nabla \times #1}

%% Fractions
\newcommand{\half}{\ensuremath{\textstyle\frac{1}{2}}}
\newcommand{\halfi}{\ensuremath{\textstyle\frac{1}{2i}}}
\newcommand{\halfpii}{\ensuremath{\textstyle\frac{1}{2\pi i}}}
\newcommand{\third}{\ensuremath{\textstyle\frac{1}{3}}}

%% Orders
\renewcommand{\th}{^{\mathrm{th}}}
\newcommand{\st}{^{\mathrm{st}}}

%% Approximation to specified order
\newcommand{\apx}[1]{\mathrel{\overset{#1}{\approx}}}

%% Riemann sums
\newcommand{\rstag}[2]{{#1}_{#2}^*}

%% Complex conjugation
\newcommand{\conj}[1]{\bar{#1}}
\newcommand{\zbar}{\conj{z}}
\newcommand{\wbar}{\conj{w}}

%% Parametrized curves
\newcommand{\curve}{C}
\newcommand{\curvept}{\mathbf{x}}
\newcommand{\dcurvept}{\dd\mathbf{x}}
\newcommand{\curvex}{x}
\newcommand{\curvey}{y}
\newcommand{\curvez}{z}

%% Regions
\newcommand{\region}{R}
\newcommand{\bdry}{\partial}

%% Parametrized surfaces
\newcommand{\surf}{S}
\newcommand{\surfx}{x}
\newcommand{\surfy}{y}
\newcommand{\surfz}{z}

%% Integrals of differential forms.  The only argument is the
%% integration manifold; the differential form just comes afterwards.
\newcommand{\lint}[1]{\int_{#1}} % line integral over a curve
\newcommand{\sint}[1]{\iint_{#1}} % surface integral form over a surface
\newcommand{\vint}[1]{\iiint_{#1}} % volume integral over a spatial region

%% "Classical" integrals.  For these the integrand is a *function*
%% (well, technically, an expression with free variables) which is
%% given as a second argument.
\newcommand{\dint}[2]{\iint_{#1} #2 \,\dA} % double integral over a plane region
\newcommand{\tint}[2]{\iiint_{#1} #2 \,\dV} % triple integral over a spatial region
\newcommand{\itint}[4]{\int_{#2}^{#3} #4 \,\dd #1} % ordinary one-variable integral over an interval
\newcommand{\itdint}[7]{\int_{#2}^{#3} \int_{#5}^{#6} #7 \,\dd #4\,\dd #1} % double iterated integral
% Can't have more than 9 parameters
%\newcommand{\ittint}[10]{\int_{#2}^{#3} \int_{#5}^{#6} \int_{#8}^{#9} #10 \,\dd #7\,\dd #4\,\dd #1} % triple iterated integral
\makeatletter
\newcommand{\ittint}[3]{\int_{#2}^{#3} \@ittint #1}
\newcommand{\@ittint}[8]{\int_{#3}^{#4} \int_{#6}^{#7} #8 \,\dd #5\,\dd #2\,\dd #1} % triple iterated integral
\makeatother

%% Textbooks
\usepackage{version}
\includeversion{notextbook}
\excludeversion{stewart}
\excludeversion{hugheshallett}
\excludeversion{rogawski}

\makeatother

\title{Line integrals}
\begin{document}
\maketitle

We introduced differential forms by saying that when you wrote
\[ \int_a^b f(x) \,\dx \]
in one-variable calculus, you were actually integrating not the function $f$ but the differential form $f(x)\,\dx$.
We will now make precise what it means to integrate a differential form, simultaneously generalizing to the multivariable situation.

\section{Integrating forms in one dimension}
\label{sec:integrating-forms}

When we ``integrate a function $f$'' in one-variable calculus from $a$ to $b$, we divide up the interval $[a,b]$ into a large number $N$ of small subintervals $[x_{i-1},x_i]$ of width $\Delta x_i$, with $x_0 = a$ and $x_N=b$, and consider the \emph{Riemann sum}
\[ \sum_{i=1}^N f(\rstag x i)\, \Delta x_i \]
where $\rstag x i$ is some point in $[x_{i-1},x_i]$.
Then we take a limit as the subinterval widths $\Delta x_i$ go to zero.

If we regard this instead as integrating the differential form $\omega(x,\dx) = f(x)\, \dx$, we can rewrite the Riemann sum as
\[ \sum_{i=1}^N \omega(\rstag x i,\Delta x_i). \]
In other words, the differential form specifies, for each value $\rstag x i$ of $x$ and each subinterval width $\Delta x_i$, the appropriate (approximate) contribution $\omega(\rstag x i,\Delta x_i)$ to the integral.
We simply add up these contributions, and then take the limit.

We can now generalizes this to integrate \emph{any} differential form in one dimension.
We simply define
\[ \int_a^b \omega = \lim_{\Delta x_i \to 0} \sum_{i=1}^N \omega(\rstag x i,\Delta x_i). \]
if this limit exists.
Note that when integrating a differential form, we do not write a ``$\dx$'' at the end of the integral: we simply write $\int_a^b \omega$.
The $\dx$ at the end of an ordinary integral is \emph{part} of the differential form being integrated.

\begin{eg}
  As a first new example, consider the form $f(x) \, |\dx|$, where $f$ is some continuous function.
  If $a<b$, so that each $\Delta x_i$ is positive, then the Riemann sums for $\int_a^b f(x) \, |\dx|$ are the same as those for $\int_a^b f(x) \, \dx$, and thus the integral is the same.
  The difference is only visible if we consider also the case when $b<a$, so that each $\Delta x_i$ is negative.
  In one-variable calculus, you probably learned that
  \[ \int_b^a f(x)\,\dx = - \int_a^b f(x) \,\dx. \]
  This is true because the Riemann sums for the first integral are the negatives of those for the second.
  However, because of the absolute value, the Riemann sums for $\int_a^b f(x) \, |\dx|$ are the \emph{same} as those for $\int_b^a f(x) \, |\dx|$; thus we have
  \[ \int_b^a f(x)\,|\dx| = \int_a^b f(x) \,|\dx|. \]
\end{eg}

\begin{eg}
  Consider the form $\dx^2$.
  If we have a tagged partition with $\Delta x_i < \epsilon$ for all $i$, then the corresponding Riemann sum is
  \begin{align*}
    \sum_{i=1}^N (\Delta x_i)^2 &< \epsilon \sum_{i=1}^N (\Delta x_i)\\
    &= \epsilon (b-a).
  \end{align*}
  In the limit, $\epsilon \to 0$, and thus the Riemann sums also go to zero.
  So we have
  \[ \int_a^b \dx^2 = 0. \]
  Intuitively, we may say that an ordinary integral $\int_a^b f(x) \,\dx$ obtains a zeroth-order result by adding up a large number of small first-order changes; but if we try to add up the \emph{same} number of \emph{second-order} changes, the result will be negligible.
\end{eg}

More generally, we have the following:

\begin{thm}\label{thm:int-gtfirstorder-onevar}
  If $\omega$ is greater than first order, then
  \[ \int_a^b \omega \;=\; 0\]
  for any closed interval $[a,b]$.
\end{thm}

In particular, it follows that
\begin{center}
  If $\omega\apx1 \eta$, then $\displaystyle\int_a^b \omega = \int_a^b \eta$.
\end{center}
For example, recall that if $f$ is differentiable, then
\begin{align*}
  f(x+\dx)-f(x) &\apx1 \df\\
  &= f'(x) \,\dx
\end{align*}
Therefore, we have
\[ \int_a^b f'(x)\,\dx = \int_a^b (f(x+\dx) - f(x)). \]
Now note that for any partition of $[a,b]$ into subintervals, we have $x_{i-1} + \dx = x_i$, by definition.
Thus, if we choose the left-hand endpoints (that is, $\rstag x i = x_{i-1}$), then the Riemann sum for $\int_a^b (f(x+\dx) - f(x))$ will be
\begin{align*}
  \Big(f(x_1) - f(x_0)\Big) + \Big(f(x_2)-f(x_1)\Big) + \cdots + \Big(f(x_N) - f(x_{N-1})\Big)
  &= f(x_N) - f(x_0)\\
  &= f(b) - f(a).
\end{align*}
Therefore, $\int_a^b f'(x)\,\dx = f(b) - f(a)$.
This is the fundamental theorem of calculus.

\begin{adv}
  The full generality of \cref{thm:int-gtfirstorder-onevar} actually requires a slightly more powerful definition of integration than is usually taught in one-variable calculus, called ``Henstock integration''.
  Roughly, we need to be more flexible with the exact meaning of what it means for the $\Delta x_i$ to all approach zero at once.
  Since we will not be making this very precise anyway, we henceforth ignore this issue.
\end{adv}

\section{Line integrals}
\label{sec:line-integrals}

If we want to integrate a form in more than one dimension, we have a new problem: we need more information to specify the ``interval'' over which to integrate.
The most appropriate sort of ``interval'' turns out to be a \textbf{parametrized curve} (or \emph{parametric curve}), a notion which you might have encountered in one-variable calculus.

Recall that a parametrized curve $\curve$ in two dimensions consists of a \emph{parameter interval} $[a,b]$ and two functions $\curvex(t)$ and $\curvey(t)$ defined for all $t\in [a,b]$.
The curve is then the set of points $\curvept(t) = (\curvex(t),\curvey(t))$ as $t$ ranges over some domain (we call $t$ the \emph{parameter}).
Similarly, in three dimensions we have three functions $\curvex(t)$, $\curvey(t)$, and $\curvez(t)$ and we consider the points $\curvept(t) = (\curvex(t),\curvey(t),\curvez(t))$.
We say that a parametrized curve is \textbf{continuous} or \textbf{differentiable} if the functions $\curvex$, $\curvey$, and (possibly) $\curvez$ are continuous or differentiable.

Now suppose $\omega$ is a differential form in two dimensions, which we want to integrate over a parametrized curve $\curve$ defined on $[a,b]$.
We regard $\omega(\pt{x},\dpx)$ as telling us about the first-order changes in some quantity Q as $\pt{x}$ changes to $\pt{x}+\dpx$, whereas we are interested in the total change of Q ``along $\curve$''.
We can approximate this total change by dividing $[a,b]$ into a large number $N$ of subintervals $[t_{i-1},t_i]$, and choosing a point $\rstag t i$ in each such subinterval.
Then if we consider the part of $\curve$ between $t_{i-1}$ and $t_i$ to be approximately a straight line, we can approximate the first-order change in Q along that part by $\omega(\curvept(\rstag t i),\Delta \curvept_i)$, where $\Delta\curvept_i = \curvept(t_i)-\curvept(t_{i-1})$.
Adding these up, we obtain the Riemann sum
\[ \sum_{i=1}^N \omega(\curvept(\rstag t i),\Delta \curvept_i) \]
and we define the integral of $\omega$ over $\curve$ to be the limit of these sums, as the widths $\Delta t_i = t_i - t_{i-1}$ go to zero:
\[ \lint{\curve} \omega = \lim_{\Delta t_i \to 0} \sum_{i=1}^N \omega(\curvept(\rstag t i),\Delta \curvept_i). \]

\subsection{Arc length}
\label{sec:arc-length}

One important example of a line integral that you may already have encountered in one-variable calculus is the \textbf{arc length} of a parametrized curve.
To find this as an integral, we should choose $\omega$ to be a first-order approximation to the length of a bit of curve.
If we consider the curve to be approximately a line, then this would be just the length of the vector $\dpx$.
Thus, in two dimensions we should integrate the differential form
\[ \sqrt{\dx^2+\dy^2}\]
while in three dimensions we should integrate
\[ \sqrt{\dx^2+\dy^2 + \dz^2}.\]
It is traditional to denote these differential forms by $\mathrm{d}\ell$, but this is potentially confusing because they are not the differential of anything.
We will instead denote them by $\dl$ (note the stroke over the $\mathrm{d}$).
That is, we define
\[ \dl = \mgn{\dpx} =
\begin{cases}
  \sqrt{\dx^2+\dy^2}\qquad\text{in two dimensions}\\[3pt]
  \sqrt{\dx^2+\dy^2+\dz^2}\qquad\text{in three dimensions}.
\end{cases}\]
Now the arc length of a curve $\curve$ is defined to be
\[ \text{arc length of } \curve = \lint{\curve} \dl \]
if this integral exists.


\end{document}
