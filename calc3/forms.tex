\documentclass[12pt]{amsart}
\usepackage{fullpage}
\makeatletter

%% AMS
\usepackage{amsmath,amssymb,latexsym}

%% Theorems
\usepackage{xcolor}
\definecolor{darkgreen}{rgb}{0,0.45,0} 
\usepackage[pagebackref,colorlinks,citecolor=darkgreen,linkcolor=darkgreen]{hyperref}
\usepackage{cleveref,aliascnt}
\usepackage{amsthm}
\def\defthm#1#2#3{%
  %% Ensure all theorem types are numbered with the same counter
  \newaliascnt{#1}{thm}
  \newtheorem{#1}[#1]{#2}
  \aliascntresetthe{#1}
  %% Tell cleveref's \cref what to call things
  \crefname{#1}{#2}{#3}}
\newtheorem{thm}{Theorem}[section]
\crefname{thm}{Theorem}{Theorems}
\theoremstyle{definition}
\defthm{defn}{Definition}{Definitions}
%\theoremstyle{remark}
\defthm{rmk}{Remark}{Remarks}
\defthm{advrmk}{Advanced Remark}{Advanced Remarks}
\newenvironment{adv}{\small\begin{advrmk}}{\end{advrmk}}
\defthm{eg}{Example}{Examples}
\defthm{egs}{Examples}{Examples}
\defthm{sage}{Sage}{Sages}

%% Reference format for sections
\crefformat{section}{\S#2#1#3}
\Crefformat{section}{Section~#2#1#3}
\crefrangeformat{section}{\S\S#3#1#4--#5#2#6}
\Crefrangeformat{section}{Sections~#3#1#4--#5#2#6}
\crefmultiformat{section}{\S\S#2#1#3}{ and~#2#1#3}{, #2#1#3}{ and~#2#1#3}
\Crefmultiformat{section}{Sections~#2#1#3}{ and~#2#1#3}{, #2#1#3}{ and~#2#1#3}
\crefrangemultiformat{section}{\S\S#3#1#4--#5#2#6}{ and~#3#1#4--#5#2#6}{, #3#1#4--#5#2#6}{ and~#3#1#4--#5#2#6}
\Crefrangemultiformat{section}{Sections~#3#1#4--#5#2#6}{ and~#3#1#4--#5#2#6}{, #3#1#4--#5#2#6}{ and~#3#1#4--#5#2#6}

%% Use the theorem counter for equations as well
\let\c@equation\c@thm
\numberwithin{equation}{section}

%% Pictures
\usepackage{tikz}
\usepackage{graphicx}

%% Differentials
\newcommand{\dd}{\ensuremath{\mathrm{d}}}
\newcommand{\ed}{\ensuremath{\mathrm{d}\wedge}}

%% Differentials of variables
\newcommand{\dx}{{\ensuremath{\dd x}}}
\newcommand{\dy}{{\ensuremath{\dd y}}}
\newcommand{\dz}{{\ensuremath{\dd z}}}
\newcommand{\dzbar}{{\ensuremath{\dd \bar{z}}}}
\newcommand{\dt}{{\ensuremath{\dd t}}}
\newcommand{\du}{{\ensuremath{\dd u}}}
\newcommand{\dv}{{\ensuremath{\dd v}}}
\newcommand{\df}{{\ensuremath{\dd f}}}
\newcommand{\dg}{{\ensuremath{\dd g}}}
\renewcommand{\dh}{{\ensuremath{\dd h}}}
\newcommand{\dr}{{\ensuremath{\dd r}}}
\newcommand{\drho}{{\ensuremath{\dd \rho}}}
\newcommand{\dtheta}{{\ensuremath{\dd \theta}}}
\newcommand{\dphi}{{\ensuremath{\dd \phi}}}
\newcommand{\dpx}{{\ensuremath{\dd \pt{x}}}}

%% Not differentials of anything
\usepackage[T1]{fontenc}
\usepackage{lmodern}
\newcommand{\dl}{\ensuremath{\mathrm{\dj}\ell}}
\newcommand{\dA}{\ensuremath{\mathrm{\dj}A}}
\newcommand{\dV}{\ensuremath{\mathrm{\dj}V}}
\newcommand{\dn}{\ensuremath{\mathrm{\dj}\vc{n}}}

%% Wedge product
%\newcommand{\wedge}{\wedge}

%% Partial derivatives
% \newcommand{\pder}[2]{\frac{\partial #1}{\partial #2}}
% \newcommand{\pdertwo}[2]{\frac{\partial^2 #1}{\partial #2^2}}
% \newcommand{\pdermixed}[3]{\frac{\partial^2 #1}{\partial #2 \,\partial #3}}
\newcommand{\pder}[2]{{#1}_{#2}}
\newcommand{\pdertwo}[2]{{#1}_{{#2}{#2}}}
\newcommand{\pdermixed}[3]{{#1}_{{#2}{#3}}}

%% Transposing vectors into forms
\newcommand{\tr}[1]{{#1}^\top}
\newcommand{\ptr}[1]{{(#1)}^\top}

%% Hodge star on forms
\newcommand{\str}[1]{{#1}^*}
\newcommand{\pstr}[1]{{(#1)}^*}

%% Points and vectors
\newcommand{\pt}[1]{\mathbf{#1}}
\newcommand{\ptc}[1]{(#1)}
\newcommand{\ptx}{\pt{x}}
% \newcommand{\vc}[1]{\vec{#1}}
\newcommand{\vc}[1]{\mathbf{#1}}
\newcommand{\vcc}[1]{(#1)}
\newcommand{\lrvcc}[1]{\left(#1\right)}

%% Vector operations
\newcommand{\mgn}[1]{\Vert #1 \Vert}
%\newcommand{\dotp}[2]{\langle #1 , #2 \rangle}
\newcommand{\dotp}[2]{#1 \cdot #2}
\newcommand{\crossp}[2]{#1 \times #2}

%% Vector calculus operations
\newcommand{\grad}[1]{\nabla #1}
\renewcommand{\div}[1]{\nabla \cdot #1}
\newcommand{\curl}[1]{\nabla \times #1}

%% Fractions
\newcommand{\half}{\ensuremath{\textstyle\frac{1}{2}}}
\newcommand{\halfi}{\ensuremath{\textstyle\frac{1}{2i}}}
\newcommand{\halfpii}{\ensuremath{\textstyle\frac{1}{2\pi i}}}
\newcommand{\third}{\ensuremath{\textstyle\frac{1}{3}}}

%% Orders
\renewcommand{\th}{^{\mathrm{th}}}
\newcommand{\st}{^{\mathrm{st}}}

%% Approximation to specified order
\newcommand{\apx}[1]{\mathrel{\overset{#1}{\approx}}}

%% Riemann sums
\newcommand{\rstag}[2]{{#1}_{#2}^*}

%% Complex conjugation
\newcommand{\conj}[1]{\bar{#1}}
\newcommand{\zbar}{\conj{z}}
\newcommand{\wbar}{\conj{w}}

%% Parametrized curves
\newcommand{\curve}{C}
\newcommand{\curvept}{\mathbf{x}}
\newcommand{\dcurvept}{\dd\mathbf{x}}
\newcommand{\curvex}{x}
\newcommand{\curvey}{y}
\newcommand{\curvez}{z}

%% Regions
\newcommand{\region}{R}
\newcommand{\bdry}{\partial}

%% Parametrized surfaces
\newcommand{\surf}{S}
\newcommand{\surfx}{x}
\newcommand{\surfy}{y}
\newcommand{\surfz}{z}

%% Integrals of differential forms.  The only argument is the
%% integration manifold; the differential form just comes afterwards.
\newcommand{\lint}[1]{\int_{#1}} % line integral over a curve
\newcommand{\sint}[1]{\iint_{#1}} % surface integral form over a surface
\newcommand{\vint}[1]{\iiint_{#1}} % volume integral over a spatial region

%% "Classical" integrals.  For these the integrand is a *function*
%% (well, technically, an expression with free variables) which is
%% given as a second argument.
\newcommand{\dint}[2]{\iint_{#1} #2 \,\dA} % double integral over a plane region
\newcommand{\tint}[2]{\iiint_{#1} #2 \,\dV} % triple integral over a spatial region
\newcommand{\itint}[4]{\int_{#2}^{#3} #4 \,\dd #1} % ordinary one-variable integral over an interval
\newcommand{\itdint}[7]{\int_{#2}^{#3} \int_{#5}^{#6} #7 \,\dd #4\,\dd #1} % double iterated integral
% Can't have more than 9 parameters
%\newcommand{\ittint}[10]{\int_{#2}^{#3} \int_{#5}^{#6} \int_{#8}^{#9} #10 \,\dd #7\,\dd #4\,\dd #1} % triple iterated integral
\makeatletter
\newcommand{\ittint}[3]{\int_{#2}^{#3} \@ittint #1}
\newcommand{\@ittint}[8]{\int_{#3}^{#4} \int_{#6}^{#7} #8 \,\dd #5\,\dd #2\,\dd #1} % triple iterated integral
\makeatother

%% Textbooks
\usepackage{version}
\includeversion{notextbook}
\excludeversion{stewart}
\excludeversion{hugheshallett}
\excludeversion{rogawski}

\makeatother

\title{On differential forms}
\begin{document}
\maketitle

In your previous calculus classes, you may or may not have encountered \emph{differential forms} by name.
However, you've certainly met them, even if you didn't realize it.
When you write
\[ \int_{x=a}^b f(x) \,\dx \]
the thing being integrated, ``$f(x) \,\dx$'', is a differential form.

You may have been told that the ``$\dx$'' is simply a notation that indicates which variable we're integrating, but this is a lie.
In multivariable calculus, we can no longer maintain this fiction: we have to treat differential forms as honest objects.
Fortunately, we also have two advantages over one-variable calculus in understanding differential forms: we have \emph{vectors} at our disposal; and some things are actually \emph{less} confusing in more dimensions because there are fewer coincidences to get confused by.

\section{Differential forms}
\label{sec:differential-forms}

If $\dx$ isn't just a notation indicating the integration variable, what is it?
If you encountered differentials in one-variable calculus, you may have been told that $\dx$ is a small change in $x$ (sometimes denoted $\Delta x$), or even an ``infinitesimal'' change in $x$.
These are not wrong, but with vectors we can give a better definition.

\begin{defn}
  A \textbf{differential 1-form} is a function whose input is a point $\pt{x}$ \emph{and} a vector $\vc{v}$ based at $\pt{x}$.
\end{defn}

In $n$ dimensions, the point $\pt{x}$ has $n$ coordinates, as does the vector $\vc{v}$; thus a differential form $\omega$ takes $2n$ inputs.
As usual, in 2 or 3 dimensions we denote the coordinates of $\pt{x}$ by $\ptc{x,y}$ or $\ptc{x,y,z}$.

We also denote the coordinates of $\vc{v}$ by $\vcc{\dx,\dy}$ or $\vcc{\dx,\dy,\dz}$, and we may write the vector $\vc{v}$ itself as $\dd\pt{x}$.
We can then describe a differential form by a formula involving $x,y,z,\dx,\dy,\dz$.

\begin{egs}
  The following are all differential 1-forms:
  \begin{gather*}
    x^2 \,\dx + 2xy \, \dy\\
    \half({x-z}) \,\dx - 4\, \dy + e^y \,\dz\\
    1 + \dx + \half \dx^2 + \third \dx^3\\
    \sin(x + \dx) - \sin(x)\\
    \sqrt{\dx^2+\dy^2}
  \end{gather*}
\end{egs}

Of these examples, the first two are special.

\begin{defn}
  A \textbf{linear} differential 1-form is one defined by an expression of the form
  \[ f(\pt{x}) \, \dx \]
  or a sum of several such expressions.
  Here $f$ is a function of the point $\pt{x}$ only, while $\dx$ is one of the coordinates of $\vc{v}$.
\end{defn}

Most mathematicians only use the term ``differential 1-form'' for linear ones.
However, the more general ones are quite useful.

It is helpful to think of the coordinates $\dx$, $\dy$, and $\dz$ as small changes in the value of $x$.
For instance, if $x=3$, then a possible small change would be $\dx = 0.001$.
In this case, a differential form such as $\dx^2$ is an \emph{even smaller} change, since $\dx^2 = 0.000001$.
It is useful to classify the \emph{order} of a differential form as follows.

\begin{defn}
  Let $k$ be an integer $\ge 0$.
  A differential 1-form $\omega$ is \textbf{$k\th$ order} if
  \[ \lim_{h\to 0} \frac{\omega(\pt{x},h\,\dd\pt{x})}{h^{k-1}} = 0 \]
  for all $\pt{x}$ and $\dd\pt{x}$.
\end{defn}

\begin{eg}
  Any linear differential 1-form is first order.
  For instance, if $\omega(\pt{x},\dd\pt{x}) = f(\pt{x}) \, \dx$, then
  \[ \frac{\omega(\pt{x},h\,\dd\pt{x})}{h^0} = f(\pt{x}) \, h\, \dx,\]
  which goes to zero as $h\to 0$ (for fixed $\pt{x}$ and $\dd\pt{x}$).
\end{eg}

\begin{eg}
  A differential 1-form not depending on $\dd\pt{x}$ (that is, essentially an ordinary function of $\pt{x}$) is zeroth order.
  Since $\frac{1}{h^{-1}} = h$, we have
  \[\frac{\omega(\pt{x},h\,\dd\pt{x})}{h^{k-1}} = h \,\omega(\pt{x}),\]
  which goes to zero as $h\to 0$.
\end{eg}

\begin{eg}
  The differential form $\dx^2$ is second order.
  We have
  \[ \frac{\omega(\pt{x},h\,\dd\pt{x})}{h^1} = \frac{(h\,\dx)^2}{h} = \frac{h^2 \, \dx^2}{h} = h \,\dx^2 \]
  which goes to zero as $h\to 0$.
\end{eg}

In general, a polynomial has the order of the lowest exponent of $\dx$ appearing in any of its terms.
Thus, $x^3\, \dx^2 + \dx^3 - 2x \,\dx^4$ is second order.

\begin{rmk}
  If a form is $k\th$ order, then it is also $(k-1)\st$ order; thus any second order form is also first order and zeroth order, and so on.
  However, a $k\th$ order form is not generally also $(k+1)\st$ order.
  We usually emphasize the greatest order that a form has; e.g.\ we almost always care most that $\dx^2$ is second order, even though it is also first and zeroth order.
\end{rmk}


\section{Differentials}
\label{sec:differentials}

In one-variable calculus, you may have encountered the differential of a function.
Namely, if $f$ is a function of one variable $x$, then its differential is
\begin{equation}
  \dd(f(x)) = f'(x) \, \dx\label{eq:onevariable-differential}
\end{equation}
where $f'$ is the \emph{derivative} of $f$.
This can be interpreted in terms of the graph of the function $f$\dots

\Cref{eq:onevariable-differential} defines the \emph{differential} of $f$ in terms of the \emph{derivative} of $f$.
However, in multivariable calculus, it is more appropriate to do things in the other order, defining the differential first.
Linear approximation\dots

\begin{defn}
  If $f$ is a function of a point $\pt{x}$, then its \textbf{differential} is a linear differential 1-form $\dd f$ such that
  \[ f(\pt{x} + \dd\pt{x}) - f(\pt{x}) - \dd f \]
  is second order.
  If such a form $\dd f$ exists, we say that $f$ is \textbf{differentiable}.
\end{defn}

\begin{eg}
  Let $f(x) = x^2$, an ordinary one-variable function.
  Then
  \begin{align*}
    f(x + \dx) - f(x) &= (x+\dx)^2 - x^2\\
    &= x^2 + 2 x \, \dx + \dx^2 - x^2\\
    &= 2x \, \dx + \dx ^2.
  \end{align*}
  What differential form $\dd f$ can we subtract from this to make it second order?
  Recalling the rule for orders of polynomials, we may think of subtracting $2x \, \dx$, leaving the second-order term $\dx^2$:
  \[ f(x + \dx) - f(x) - 2x \, \dx \;=\; \dx^2. \]
  Thus, we have $\dd(x^2) = 2 x \, \dx$.
  Of course, this is the same result that we would have gotten from \cref{eq:onevariable-differential}.
\end{eg}

All the usual derivative rules of one-variable calculus can be derived from this definition as well.\dots

We can now define the \emph{derivative} in terms of the \emph{differential}.
Note that in one dimension, a linear differential 1-form must be of the form
\[ g(x) \,\dx \]
for some function $g$.
In particular, if $f$ is a differentiable function, then $\dd f = g(x) \, \dx$ for some function $g$, which we call the \emph{derivative} of $f$ and denote by $f'$.
In other words,
\[ f'(x) = \frac{\dd f}{\dx}, \]
which is a notation that you have probably seen before.
The point is that now, we have given separate meanings to $\dx$ and $\dd f$, and \emph{defined} $f'(x)$ to be their quotient.


\end{document}