\documentclass{amsart}
\makeatletter

%% Theorems
\usepackage{xcolor}
\definecolor{darkgreen}{rgb}{0,0.45,0} 
\usepackage[pagebackref,colorlinks,citecolor=darkgreen,linkcolor=darkgreen]{hyperref}
\usepackage{cleveref,aliascnt}
\def\defthm#1#2#3{%
  %% Ensure all theorem types are numbered with the same counter
  \newaliascnt{#1}{thm}
  \newtheorem{#1}[#1]{#2}
  \aliascntresetthe{#1}
  %% Tell cleveref's \cref what to call things
  \crefname{#1}{#2}{#3}}
\newtheorem{thm}{Theorem}[section]
\crefname{thm}{Theorem}{Theorems}
\theoremstyle{definition}
\defthm{defn}{Definition}{Definitions}
%\theoremstyle{remark}
\defthm{rmk}{Remark}{Remarks}
\defthm{advrmk}{Advanced Remark}{Advanced Remarks}
\newenvironment{adv}{\SMALL\begin{advrmk}}{\end{advrmk}}
\defthm{eg}{Example}{Examples}
\defthm{egs}{Examples}{Examples}

%% Reference format for sections
\crefformat{section}{\S#2#1#3}
\Crefformat{section}{Section~#2#1#3}
\crefrangeformat{section}{\S\S#3#1#4--#5#2#6}
\Crefrangeformat{section}{Sections~#3#1#4--#5#2#6}
\crefmultiformat{section}{\S\S#2#1#3}{ and~#2#1#3}{, #2#1#3}{ and~#2#1#3}
\Crefmultiformat{section}{Sections~#2#1#3}{ and~#2#1#3}{, #2#1#3}{ and~#2#1#3}
\crefrangemultiformat{section}{\S\S#3#1#4--#5#2#6}{ and~#3#1#4--#5#2#6}{, #3#1#4--#5#2#6}{ and~#3#1#4--#5#2#6}
\Crefrangemultiformat{section}{Sections~#3#1#4--#5#2#6}{ and~#3#1#4--#5#2#6}{, #3#1#4--#5#2#6}{ and~#3#1#4--#5#2#6}

%% Use the theorem counter for equations as well
\let\c@equation\c@thm
\numberwithin{equation}{section}

%% Pictures
\usepackage{tikz}

%% Differentials
\newcommand{\dd}{\ensuremath{\mathrm{d}}}
\newcommand{\dx}{\ensuremath{\dd x}}
\newcommand{\dy}{\ensuremath{\dd y}}
\newcommand{\dz}{\ensuremath{\dd z}}
\newcommand{\dt}{\ensuremath{\dd t}}
\newcommand{\du}{\ensuremath{\dd u}}
\newcommand{\dv}{\ensuremath{\dd v}}
\newcommand{\df}{\ensuremath{\dd f}}
\newcommand{\dg}{\ensuremath{\dd g}}
\newcommand{\dr}{\ensuremath{\dd r}}
\newcommand{\drho}{\ensuremath{\dd \rho}}
\newcommand{\dtheta}{\ensuremath{\dd \theta}}
\newcommand{\dphi}{\ensuremath{\dd \phi}}
\newcommand{\dpx}{\ensuremath{\dd \pt{x}}}

%% Partial derivatives
\newcommand{\pder}[2]{\frac{\partial #1}{\partial #2}}

%% Transposing vectors into forms
\newcommand{\tr}[1]{{#1}^\top}
\newcommand{\ptr}[1]{{(#1)}^\top}

%% Hodge star on forms
\newcommand{\str}[1]{{#1}^*}
\newcommand{\pstr}[1]{{(#1)}^*}

%% Points and vectors
\newcommand{\pt}[1]{\mathbf{#1}}
\newcommand{\ptc}[1]{(#1)}
\newcommand{\vc}[1]{\vec{#1}}
\newcommand{\vcc}[1]{(#1)}

%% Vector operations
\newcommand{\mgn}[1]{\Vert #1 \Vert}
\newcommand{\dotp}[2]{\langle #1 , #2 \rangle}
\newcommand{\crossp}[2]{#1 \times #2}

%% Fractions
\newcommand{\half}{\ensuremath{\textstyle\frac{1}{2}}}
\newcommand{\third}{\ensuremath{\textstyle\frac{1}{3}}}

%% Orders
\renewcommand{\th}{^{\mathrm{th}}}
\newcommand{\st}{^{\mathrm{st}}}

%% Approximation to specified order
\newcommand{\apx}[1]{\mathrel{\overset{#1}{\approx}}}

\makeatother


\begin{document}

\begin{defn}\label{def:order}
  Let $n$ be an integer $\ge 0$.
  A differential form $\omega$ is \textbf{$n\th$ order} at a point $\ptx$ if
  \[ \frac{\omega(\ptx,\dpx)}{\mgn{\dpx}^n} \]
  remains bounded as $\dpx\to\vc{0}$.  In other words, there is a number $M>0$ such that
  \( \frac{\omega(\ptx,\dpx)}{\mgn{\dpx}^n} \le M \)
  for all $\dpx$ sufficiently close to $\vc{0}$.
  The property of $\omega$ being $n\th$ order is also written as
  \[ \omega \in O(\dpx^n) \quad\text{as}\;\dpx\to\vc{0} \]
  which is pronounced as ``$\omega$ is big-oh of $\dpx^n$ as $\dpx\to\vc{0}$''.
\end{defn}

\begin{eg}
  As suggested above, any \emph{linear} differential form is first order at every point.
  For instance, if $\omega(\ptx,\dd\ptx) = f(\ptx) \, \dx$, then
  \[ \frac{\omega(\ptx,\dpx)}{\mgn{\dpx}^1} = f(\ptx) \frac{\dx}{\mgn{\dpx}}.\]
  Since $f(\ptx)$ does not depend on $\dpx$, and $\frac{\dx}{\mgn{\dpx}} \le 1$, this is bounded by $M=f(\dpx)$ as $\dpx\to\vc0$.
\end{eg}

\begin{eg}
  \emph{Any} continuous differential form is zeroth order at every point.
  For we have
  \[\frac{\omega(\ptx,\dpx)}{\mgn{\dpx}^0} = \omega(\ptx,\dpx).\]
  Since $\omega$ is continuous, $\lim_{\dpx\to 0} \omega(\ptx,\dpx) = \omega(\ptx,\vc{0})$, and in particular is finite.
  Thus, in particular $\omega(\ptx,\dpx)$ is bounded.
\end{eg}

\begin{eg}
  The differential form $\dx^2$ is second order at any point.
  We have
  \[ \frac{\omega(\ptx,\dpx)}{\mgn{\dpx}^2} = \frac{\dx^2}{\dx^2} = 1 \]
  which is constant, hence bounded as $\dpx\to\vc0$.
\end{eg}

\begin{eg}
  The differential form $\sqrt{\dx^2+\dy^2}$ is first order at any point.
  We have
  \[ \frac{\sqrt{\dx^2+\dy^2}}{\mgn{\dpx}^1} =
  \frac{\sqrt{\dx^2+\dy^2}}{\sqrt{\dx^2+\dy^2}} = 1
  \]
  which is clearly bounded.
\end{eg}

We also have the following rules for orders of forms.

\begin{thm}
  Let $\omega$ and $\eta$ be differential forms, and assume that $\omega$ is $n\th$ order and $\eta$ is $m\th$ order at $\ptx$.
  \begin{enumerate}
  \item $\omega+\eta$ is $\min(n,m)\th$ order at $\ptx$.
  \item $\omega\eta$ is $(n+m)\th$ order at $\ptx$.
  \end{enumerate}
\end{thm}

It follows that a polynomial in the differential variables $\dx,\dy,\dz$ has the order of the lowest total degree of these variables in all of its terms.
Thus, $x^3\, \dx^2 + \dx^3 - 2x \,\dx^4$ is second order at every point, since the lowest exponent of $\dx$ appearing in all its terms is 2 (the exponents of $x$ don't matter).
Similarly, $xy\,\dx\,\dy^2 + y\, \dx^4$ is third order, since $\dx\,\dy^2$ has total degree 3 and $\dx^4$ has total degree 4.
The ``coefficients'' can be arbitrary functions of $\ptx$ as well; for instance, $e^x \,\dx^2$ is second order.

\begin{rmk}
  If a form is $n\th$ order, then it is also $m\th$ order for any $m<n$.
  Thus, any second order form is also first order and zeroth order, and so on.
  We usually emphasize the greatest order that a form has; e.g.\ we almost always care most that $\dx^2$ is second order, even though it is also first and zeroth order.
\end{rmk}

\begin{eg}
  A differential form can have a higher order at some points than others.
  For instance, the form $x \,\dx + \dx^2$ is first-order everywhere, but at the point $x=0$ it becomes $0\, dx + \dx^2 = \dx^2$, which is second order.
\end{eg}

\begin{rmk}
  Note that the notion of ``order'' only applies to differential forms, which are \emph{functions}.
  Although we motivated it by talking about numbers like $0.00001$ being much smaller than $0.001$, strictly speaking it does not make sense to say that a \emph{number} is first-order or second-order.
  This is because ``smallness'' is always relative to the numbers that come up in a particular problem.

  For instance, in some problem it might happen that $x=0.001$, or even $0.00001$, but the function $f(x) = x$ would still be zeroth order.
  In that problem, we would expect $\dx$ to be even smaller than $x$, such as $0.000000001$.
  On the other hand, it some other problem it might happen that $x = 10000$, in which case we might have $\dx = 10$, but the differential form $\dx$ would still be first order.

  It is possible to give a meaning to ``first order number'' by introducing infinitesimals.
  However, we will not do that.
\end{rmk}

When we define differentials, we will also need the following closely related notion.

\begin{defn}
  A differential form $\omega$ is \textbf{greater than $k\th$ order} at a point $\ptx$ if
  \[ \lim_{\dpx\to\vc{0}} \frac{\omega(\ptx,\dpx)}{\mgn{\dpx}^k} = 0. \]
  This property is also written as
  \[ \omega \in o(\dpx^k) \quad\text{as}\;\dpx\to\vc{0} \]
  which is pronounced as ``$\omega$ is little-oh of $\dpx^k$ as $\dpx\to\vc{0}$''.
\end{defn}

Unsurprisingly, if $\omega$ is $n\th$ order, then it is greater than $k\th$ order whenever $n>k$.
For in this case we have
\[\frac{\omega(\ptx,\dpx)}{\mgn{\dpx}^k} = \frac{\omega(\ptx,\dpx)}{\mgn{\dpx}^n} \cdot \mgn{\dpx}^{n-k}, \]
and the first term remains bounded while the second goes to zero if $n>k$.

For the cases that arise in practice, $n$ and $k$ are usually integers, and most forms that are greater than $k\th$ order are in fact $(k+1)\st$ order.
However, technically the two are distinct.
For instance, the form $\dx^{3/2}$ is greater than first order, but is not second order (it is $(3/2)\th$ order).
Even more interestingly, the form $\frac{1}{\ln\dx}$ is greater than zeroth order, but is not $n\th$ order for any $n>0$.

\end{document}