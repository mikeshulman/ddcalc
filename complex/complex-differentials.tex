\documentclass[12pt]{amsart}
\usepackage{fullpage}
\makeatletter

%% AMS
\usepackage{amsmath,amssymb,latexsym}

%% Theorems
\usepackage{xcolor}
\definecolor{darkgreen}{rgb}{0,0.45,0} 
\usepackage[pagebackref,colorlinks,citecolor=darkgreen,linkcolor=darkgreen]{hyperref}
\usepackage{cleveref,aliascnt}
\usepackage{amsthm}
\def\defthm#1#2#3{%
  %% Ensure all theorem types are numbered with the same counter
  \newaliascnt{#1}{thm}
  \newtheorem{#1}[#1]{#2}
  \aliascntresetthe{#1}
  %% Tell cleveref's \cref what to call things
  \crefname{#1}{#2}{#3}}
\newtheorem{thm}{Theorem}[section]
\crefname{thm}{Theorem}{Theorems}
\theoremstyle{definition}
\defthm{defn}{Definition}{Definitions}
%\theoremstyle{remark}
\defthm{rmk}{Remark}{Remarks}
\defthm{advrmk}{Advanced Remark}{Advanced Remarks}
\newenvironment{adv}{\small\begin{advrmk}}{\end{advrmk}}
\defthm{eg}{Example}{Examples}
\defthm{egs}{Examples}{Examples}
\defthm{sage}{Sage}{Sages}

%% Reference format for sections
\crefformat{section}{\S#2#1#3}
\Crefformat{section}{Section~#2#1#3}
\crefrangeformat{section}{\S\S#3#1#4--#5#2#6}
\Crefrangeformat{section}{Sections~#3#1#4--#5#2#6}
\crefmultiformat{section}{\S\S#2#1#3}{ and~#2#1#3}{, #2#1#3}{ and~#2#1#3}
\Crefmultiformat{section}{Sections~#2#1#3}{ and~#2#1#3}{, #2#1#3}{ and~#2#1#3}
\crefrangemultiformat{section}{\S\S#3#1#4--#5#2#6}{ and~#3#1#4--#5#2#6}{, #3#1#4--#5#2#6}{ and~#3#1#4--#5#2#6}
\Crefrangemultiformat{section}{Sections~#3#1#4--#5#2#6}{ and~#3#1#4--#5#2#6}{, #3#1#4--#5#2#6}{ and~#3#1#4--#5#2#6}

%% Use the theorem counter for equations as well
\let\c@equation\c@thm
\numberwithin{equation}{section}

%% Pictures
\usepackage{tikz}
\usepackage{graphicx}

%% Differentials
\newcommand{\dd}{\ensuremath{\mathrm{d}}}
\newcommand{\ed}{\ensuremath{\mathrm{d}\wedge}}

%% Differentials of variables
\newcommand{\dx}{{\ensuremath{\dd x}}}
\newcommand{\dy}{{\ensuremath{\dd y}}}
\newcommand{\dz}{{\ensuremath{\dd z}}}
\newcommand{\dzbar}{{\ensuremath{\dd \bar{z}}}}
\newcommand{\dt}{{\ensuremath{\dd t}}}
\newcommand{\du}{{\ensuremath{\dd u}}}
\newcommand{\dv}{{\ensuremath{\dd v}}}
\newcommand{\df}{{\ensuremath{\dd f}}}
\newcommand{\dg}{{\ensuremath{\dd g}}}
\renewcommand{\dh}{{\ensuremath{\dd h}}}
\newcommand{\dr}{{\ensuremath{\dd r}}}
\newcommand{\drho}{{\ensuremath{\dd \rho}}}
\newcommand{\dtheta}{{\ensuremath{\dd \theta}}}
\newcommand{\dphi}{{\ensuremath{\dd \phi}}}
\newcommand{\dpx}{{\ensuremath{\dd \pt{x}}}}

%% Not differentials of anything
\usepackage[T1]{fontenc}
\usepackage{lmodern}
\newcommand{\dl}{\ensuremath{\mathrm{\dj}\ell}}
\newcommand{\dA}{\ensuremath{\mathrm{\dj}A}}
\newcommand{\dV}{\ensuremath{\mathrm{\dj}V}}
\newcommand{\dn}{\ensuremath{\mathrm{\dj}\vc{n}}}

%% Wedge product
%\newcommand{\wedge}{\wedge}

%% Partial derivatives
% \newcommand{\pder}[2]{\frac{\partial #1}{\partial #2}}
% \newcommand{\pdertwo}[2]{\frac{\partial^2 #1}{\partial #2^2}}
% \newcommand{\pdermixed}[3]{\frac{\partial^2 #1}{\partial #2 \,\partial #3}}
\newcommand{\pder}[2]{{#1}_{#2}}
\newcommand{\pdertwo}[2]{{#1}_{{#2}{#2}}}
\newcommand{\pdermixed}[3]{{#1}_{{#2}{#3}}}

%% Transposing vectors into forms
\newcommand{\tr}[1]{{#1}^\top}
\newcommand{\ptr}[1]{{(#1)}^\top}

%% Hodge star on forms
\newcommand{\str}[1]{{#1}^*}
\newcommand{\pstr}[1]{{(#1)}^*}

%% Points and vectors
\newcommand{\pt}[1]{\mathbf{#1}}
\newcommand{\ptc}[1]{(#1)}
\newcommand{\ptx}{\pt{x}}
% \newcommand{\vc}[1]{\vec{#1}}
\newcommand{\vc}[1]{\mathbf{#1}}
\newcommand{\vcc}[1]{(#1)}
\newcommand{\lrvcc}[1]{\left(#1\right)}

%% Vector operations
\newcommand{\mgn}[1]{\Vert #1 \Vert}
%\newcommand{\dotp}[2]{\langle #1 , #2 \rangle}
\newcommand{\dotp}[2]{#1 \cdot #2}
\newcommand{\crossp}[2]{#1 \times #2}

%% Vector calculus operations
\newcommand{\grad}[1]{\nabla #1}
\renewcommand{\div}[1]{\nabla \cdot #1}
\newcommand{\curl}[1]{\nabla \times #1}

%% Fractions
\newcommand{\half}{\ensuremath{\textstyle\frac{1}{2}}}
\newcommand{\halfi}{\ensuremath{\textstyle\frac{1}{2i}}}
\newcommand{\halfpii}{\ensuremath{\textstyle\frac{1}{2\pi i}}}
\newcommand{\third}{\ensuremath{\textstyle\frac{1}{3}}}

%% Orders
\renewcommand{\th}{^{\mathrm{th}}}
\newcommand{\st}{^{\mathrm{st}}}

%% Approximation to specified order
\newcommand{\apx}[1]{\mathrel{\overset{#1}{\approx}}}

%% Riemann sums
\newcommand{\rstag}[2]{{#1}_{#2}^*}

%% Complex conjugation
\newcommand{\conj}[1]{\bar{#1}}
\newcommand{\zbar}{\conj{z}}
\newcommand{\wbar}{\conj{w}}

%% Parametrized curves
\newcommand{\curve}{C}
\newcommand{\curvept}{\mathbf{x}}
\newcommand{\dcurvept}{\dd\mathbf{x}}
\newcommand{\curvex}{x}
\newcommand{\curvey}{y}
\newcommand{\curvez}{z}

%% Regions
\newcommand{\region}{R}
\newcommand{\bdry}{\partial}

%% Parametrized surfaces
\newcommand{\surf}{S}
\newcommand{\surfx}{x}
\newcommand{\surfy}{y}
\newcommand{\surfz}{z}

%% Integrals of differential forms.  The only argument is the
%% integration manifold; the differential form just comes afterwards.
\newcommand{\lint}[1]{\int_{#1}} % line integral over a curve
\newcommand{\sint}[1]{\iint_{#1}} % surface integral form over a surface
\newcommand{\vint}[1]{\iiint_{#1}} % volume integral over a spatial region

%% "Classical" integrals.  For these the integrand is a *function*
%% (well, technically, an expression with free variables) which is
%% given as a second argument.
\newcommand{\dint}[2]{\iint_{#1} #2 \,\dA} % double integral over a plane region
\newcommand{\tint}[2]{\iiint_{#1} #2 \,\dV} % triple integral over a spatial region
\newcommand{\itint}[4]{\int_{#2}^{#3} #4 \,\dd #1} % ordinary one-variable integral over an interval
\newcommand{\itdint}[7]{\int_{#2}^{#3} \int_{#5}^{#6} #7 \,\dd #4\,\dd #1} % double iterated integral
% Can't have more than 9 parameters
%\newcommand{\ittint}[10]{\int_{#2}^{#3} \int_{#5}^{#6} \int_{#8}^{#9} #10 \,\dd #7\,\dd #4\,\dd #1} % triple iterated integral
\makeatletter
\newcommand{\ittint}[3]{\int_{#2}^{#3} \@ittint #1}
\newcommand{\@ittint}[8]{\int_{#3}^{#4} \int_{#6}^{#7} #8 \,\dd #5\,\dd #2\,\dd #1} % triple iterated integral
\makeatother

%% Textbooks
\usepackage{version}
\includeversion{notextbook}
\excludeversion{stewart}
\excludeversion{hugheshallett}
\excludeversion{rogawski}

\makeatother

\title{Complex differential forms}
\begin{document}
\maketitle

So far, we have discussed only \emph{real} differential forms.
We now apply these to the study of complex numbers and complex differentials.

\section{Complex differential forms}
\label{sec:complex-forms}

There is no problem with allowing the components of a differential form to be \emph{complex-valued}.
Thus, for instance, we may consider the differential 1-form $e^{ix} \,\dx + (y-ix^2)\,\dy$, or the differential 2-form $(x^2- i y^2)\,\dx\wedge\dy$.
Note that the ambient space is still (one-, two-, or three-dimensional) \emph{real} space, with the coordinates such as $x$ and $y$ being real numbers.

However, let us now restrict to considering such forms in \emph{two} dimensions, with real coordinates $x$ and $y$.
In this case, as we have seen, we can also identify the plane with the set of complex numbers, with complex coordinate $z = x+iy$.
If we regard $z$ as a complex-valued function of the two real coordinates $x$ and $y$, i.e.\ $z(x,y) = x+iy$, then its differential is a complex-valued 1-form:
\[ \dz = \dx + i \,\dy. \]
Similarly, its conjugate $\zbar$ is also a complex-valued function of $x$ and $y$, namely $\zbar(x,y) = x-iy$, and its differential is
\[ \dzbar = \dx - i\, \dy. \]

Now note that just as
\[ x = \half(z+\zbar) \qquad\text{and}\qquad y = \half(z-\zbar) \]
we have
\[ \dx = \half(\dz+\dzbar) \qquad\text{and}\qquad \dy = \halfi(\dz-\dzbar) \]
Therefore, any complex-valued differential form can be written in terms of $\dz$ and $\dzbar$ instead of $\dx$ and $\dy$.
Specifically, if we have a 1-form $f(x,y) \,\dx + g(x,y) \,\dy$, then we can write it as
\begin{align*}
  f(x,y) \,\dx + g(x,y) \,\dy
  &= f(x,y)\cdot\half(\dz+\dzbar) +  g(x,y) \cdot\halfi(\dz-\dzbar)\\
  &= \half (f(x,y) -i\, g(x,y))\,\dz + \half (f(x,y) +i\, g(x,y))\,\dzbar.
\end{align*}
We also have
\begin{align*}
  \dz \wedge\dzbar &= (\dx + i \,\dy) \wedge (\dx - i \,\dy)\\
  &= \dx\wedge\dx + i\,\dy\wedge\dx - i\,\dx\wedge\dy + \dy\wedge\dy\\
  &= -2i\, \dx\wedge\dy.
\end{align*}
Therefore, a 2-form $A\,\dx\wedge\dy$ can equivalently be written as $-\halfi A\, \dz\wedge\dzbar$.

\section{Complex differentials}
\label{sec:complex-differentials}

Now let us consider the differential as it acts on complex forms.
Suppose $f$ is a complex-valued 0-form, i.e.\ a function.
Then we have
\begin{align*}
  \df &= \pder f x \, \dx + \pder f y\,\dy\\
  &= \half (\pder f x - i\, \pder f y)\,\dz + \half (\pder f x + i\, \pder f y)\,\dzbar.
\end{align*}
Therefore, it is natural to introduce the following definitions of ``partial derivatives with respect to $z$ and $\zbar$'':
\begin{align*}
  \pder f z &= \half (\pder f x - i\, \pder f y)\\
  \pder f \zbar &= \half (\pder f x + i\, \pder f y)
\end{align*}
Thus, we can write $\df = \pder f z \,\dz + \pder f \zbar \,\dzbar$.

Let us now express these in components, writing $f(x,y) = u(x,y) + i\,v(x,y)$ for real-valued functions $u$ and $v$.
Then we have
\begin{align*}
  \pder f x &= \pder u x + i\, \pder v x\\
  \pder f y &= \pder u y + i\, \pder v y
\end{align*}
and therefore
\begin{align*}
  \pder f z &= \half (\pder u x + \pder v y) + \half i\, (\pder v x - \pder u y)\\
  \pder f \zbar &= \half (\pder u x - \pder v y) + \half i\, (\pder v x + \pder u y)
\end{align*}

Now recall the \emph{Cauchy--Riemann equations} say that if $f$ is complex-differentiable (at a point), then
\[ \pder f y = i\,\pder f x \qquad\text{or equivalently}\qquad \pder f x = - i\,\pder f y,\]
(at that point) and conversely as long as $\pder f x$ and $\pder f y$ are continuous.
In components, this equation becomes
\[ \pder u x = \pder v y \qquad\text{and}\qquad \pder u y = - \pder v x. \]
In terms of the notations introduced above, these equations say exactly that
\[ \pder f \zbar = 0. \]
Thus, intuitively, for $f$ to be complex-differentiable means exactly that it is ``constant with respect to $\zbar$'' or ``a function of $z$ only''.

Finally, the equation $\pder f y = i\,\pder f x$ also implies that
\[ \pder f z = \half(\pder f x - i\,\pder f y) = \half (\pder f x + \pder f x) = \pder f x.\]
This is also the \emph{complex derivative} $f'(z)$ of $f$.
Thus, if $f$ is complex-differentiable, then we can write
\[ \df = f'(z) \,\dz \]
which is what we would expect to see by treating $z = x+i\,y$ as a \emph{single complex variable} rather than two real variables.

\section{Complex 1-forms}
\label{sec:complex-1-forms}

As shown above, a general complex-valued 1-form can be written as
\[f(x,y) \,\dz + g(x,y)\,\dzbar\]
for some complex-valued functions $f$ and $g$.
However, the above discussion of differentials, and the general philosophy of treating $z=x+i\,y$ as a single complex variable, suggest that we should really restrict our attention to 1-forms of the form
\[ f(z)\,\dz \]
for which $f$ is complex differentiable (in some region), i.e. $\pder f \zbar = 0$.
We call these \textbf{analytic 1-forms} (in that region).
They have the interesting property that \emph{their} differentials are always zero!
\begin{align*}
  \ed(f(z)\,\dz) &= \df \wedge \dz\\
  &= f'(z)\, \dz \wedge \dz\\
  &= 0
\end{align*}
As we will see, this is crucial to the amazing properties of complex-differentiable functions.

Note that there is no meaningful notion of ``analytic 2-form''.
This is in keeping with the idea that the complex plane has only ``one complex dimension'', even though it has two real dimensions.

\end{document}
